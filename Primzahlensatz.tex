\documentclass{mywork}
\addtolength{\headheight}{\baselineskip}
\lhead{Proseminar - Funktionentheorie \\ \today}
\chead{Newmans Beweis des Primzahlensatzes}
\rhead{\theauthor}
\renewcommand{\theta}{\vartheta}
\cfoot[LE,RO]{\bfseries\color{gray} -~\thepage~-}
\usepackage{pgfplots}
\begin{document}
\section*{Der Primzahlensatz}
Zunächst einige Definitionen. Es sei angemerkt, dass $ p $ stets Element aus der Menge aller Primzahlen sein soll.

\begin{df}[Primzahlen zählende Funktion]
Sei $ x\in \R $, dann definieren wir die \emph{Primzahlen zählende Funktion} durch
\[
\pi(x):=\Big|\{p\in \mathbb P| p \le x\}\Big|=\sum_{p \in \mathbb P,\; p \le x} 1,
\]
wobei $ \mathbb P $ die Menge aller Primzahlen bezeichnet.
\end{df} 
\begin{figure}[H]
	\centering
	\newlength\figureheight 
	\newlength\figurewidth 
	\setlength\figureheight{6cm} 
	\setlength\figurewidth{10cm}
	% This file was created by matlab2tikz v0.3.3.
% Copyright (c) 2008--2013, Nico Schlömer <nico.schloemer@gmail.com>
% All rights reserved.
% 
% The latest updates can be retrieved from
%   http://www.mathworks.com/matlabcentral/fileexchange/22022-matlab2tikz
% where you can also make suggestions and rate matlab2tikz.
% 
% 
% 
\begin{tikzpicture}

\begin{axis}[%
width=\figurewidth,
height=\figureheight,
scale only axis,
xmin=0,
xmax=100,
ymin=0,
ymax=25
]
\addplot [
color=blue,
solid,
forget plot
]
table[row sep=crcr]{
0 0\\
0.1 0\\
0.2 0\\
0.3 0\\
0.4 0\\
0.5 0\\
0.6 0\\
0.7 0\\
0.8 0\\
0.9 0\\
1 0\\
1.1 0\\
1.2 0\\
1.3 0\\
1.4 0\\
1.5 0\\
1.6 0\\
1.7 0\\
1.8 0\\
1.9 0\\
2 0\\
2.1 1\\
2.2 1\\
2.3 1\\
2.4 1\\
2.5 1\\
2.6 1\\
2.7 1\\
2.8 1\\
2.9 1\\
3 1\\
3.1 2\\
3.2 2\\
3.3 2\\
3.4 2\\
3.5 2\\
3.6 2\\
3.7 2\\
3.8 2\\
3.9 2\\
4 2\\
4.1 2\\
4.2 2\\
4.3 2\\
4.4 2\\
4.5 2\\
4.6 2\\
4.7 2\\
4.8 2\\
4.9 2\\
5 2\\
5.1 3\\
5.2 3\\
5.3 3\\
5.4 3\\
5.5 3\\
5.6 3\\
5.7 3\\
5.8 3\\
5.9 3\\
6 3\\
6.1 3\\
6.2 3\\
6.3 3\\
6.4 3\\
6.5 3\\
6.6 3\\
6.7 3\\
6.8 3\\
6.9 3\\
7 3\\
7.1 4\\
7.2 4\\
7.3 4\\
7.4 4\\
7.5 4\\
7.6 4\\
7.7 4\\
7.8 4\\
7.9 4\\
8 4\\
8.1 4\\
8.2 4\\
8.3 4\\
8.4 4\\
8.5 4\\
8.6 4\\
8.7 4\\
8.8 4\\
8.9 4\\
9 4\\
9.1 4\\
9.2 4\\
9.3 4\\
9.4 4\\
9.5 4\\
9.6 4\\
9.7 4\\
9.8 4\\
9.9 4\\
10 4\\
10.1 4\\
10.2 4\\
10.3 4\\
10.4 4\\
10.5 4\\
10.6 4\\
10.7 4\\
10.8 4\\
10.9 4\\
11 4\\
11.1 5\\
11.2 5\\
11.3 5\\
11.4 5\\
11.5 5\\
11.6 5\\
11.7 5\\
11.8 5\\
11.9 5\\
12 5\\
12.1 5\\
12.2 5\\
12.3 5\\
12.4 5\\
12.5 5\\
12.6 5\\
12.7 5\\
12.8 5\\
12.9 5\\
13 5\\
13.1 6\\
13.2 6\\
13.3 6\\
13.4 6\\
13.5 6\\
13.6 6\\
13.7 6\\
13.8 6\\
13.9 6\\
14 6\\
14.1 6\\
14.2 6\\
14.3 6\\
14.4 6\\
14.5 6\\
14.6 6\\
14.7 6\\
14.8 6\\
14.9 6\\
15 6\\
15.1 6\\
15.2 6\\
15.3 6\\
15.4 6\\
15.5 6\\
15.6 6\\
15.7 6\\
15.8 6\\
15.9 6\\
16 6\\
16.1 6\\
16.2 6\\
16.3 6\\
16.4 6\\
16.5 6\\
16.6 6\\
16.7 6\\
16.8 6\\
16.9 6\\
17 6\\
17.1 7\\
17.2 7\\
17.3 7\\
17.4 7\\
17.5 7\\
17.6 7\\
17.7 7\\
17.8 7\\
17.9 7\\
18 7\\
18.1 7\\
18.2 7\\
18.3 7\\
18.4 7\\
18.5 7\\
18.6 7\\
18.7 7\\
18.8 7\\
18.9 7\\
19 7\\
19.1 8\\
19.2 8\\
19.3 8\\
19.4 8\\
19.5 8\\
19.6 8\\
19.7 8\\
19.8 8\\
19.9 8\\
20 8\\
20.1 8\\
20.2 8\\
20.3 8\\
20.4 8\\
20.5 8\\
20.6 8\\
20.7 8\\
20.8 8\\
20.9 8\\
21 8\\
21.1 8\\
21.2 8\\
21.3 8\\
21.4 8\\
21.5 8\\
21.6 8\\
21.7 8\\
21.8 8\\
21.9 8\\
22 8\\
22.1 8\\
22.2 8\\
22.3 8\\
22.4 8\\
22.5 8\\
22.6 8\\
22.7 8\\
22.8 8\\
22.9 8\\
23 8\\
23.1 9\\
23.2 9\\
23.3 9\\
23.4 9\\
23.5 9\\
23.6 9\\
23.7 9\\
23.8 9\\
23.9 9\\
24 9\\
24.1 9\\
24.2 9\\
24.3 9\\
24.4 9\\
24.5 9\\
24.6 9\\
24.7 9\\
24.8 9\\
24.9 9\\
25 9\\
25.1 9\\
25.2 9\\
25.3 9\\
25.4 9\\
25.5 9\\
25.6 9\\
25.7 9\\
25.8 9\\
25.9 9\\
26 9\\
26.1 9\\
26.2 9\\
26.3 9\\
26.4 9\\
26.5 9\\
26.6 9\\
26.7 9\\
26.8 9\\
26.9 9\\
27 9\\
27.1 9\\
27.2 9\\
27.3 9\\
27.4 9\\
27.5 9\\
27.6 9\\
27.7 9\\
27.8 9\\
27.9 9\\
28 9\\
28.1 9\\
28.2 9\\
28.3 9\\
28.4 9\\
28.5 9\\
28.6 9\\
28.7 9\\
28.8 9\\
28.9 9\\
29 9\\
29.1 10\\
29.2 10\\
29.3 10\\
29.4 10\\
29.5 10\\
29.6 10\\
29.7 10\\
29.8 10\\
29.9 10\\
30 10\\
30.1 10\\
30.2 10\\
30.3 10\\
30.4 10\\
30.5 10\\
30.6 10\\
30.7 10\\
30.8 10\\
30.9 10\\
31 10\\
31.1 11\\
31.2 11\\
31.3 11\\
31.4 11\\
31.5 11\\
31.6 11\\
31.7 11\\
31.8 11\\
31.9 11\\
32 11\\
32.1 11\\
32.2 11\\
32.3 11\\
32.4 11\\
32.5 11\\
32.6 11\\
32.7 11\\
32.8 11\\
32.9 11\\
33 11\\
33.1 11\\
33.2 11\\
33.3 11\\
33.4 11\\
33.5 11\\
33.6 11\\
33.7 11\\
33.8 11\\
33.9 11\\
34 11\\
34.1 11\\
34.2 11\\
34.3 11\\
34.4 11\\
34.5 11\\
34.6 11\\
34.7 11\\
34.8 11\\
34.9 11\\
35 11\\
35.1 11\\
35.2 11\\
35.3 11\\
35.4 11\\
35.5 11\\
35.6 11\\
35.7 11\\
35.8 11\\
35.9 11\\
36 11\\
36.1 11\\
36.2 11\\
36.3 11\\
36.4 11\\
36.5 11\\
36.6 11\\
36.7 11\\
36.8 11\\
36.9 11\\
37 11\\
37.1 12\\
37.2 12\\
37.3 12\\
37.4 12\\
37.5 12\\
37.6 12\\
37.7 12\\
37.8 12\\
37.9 12\\
38 12\\
38.1 12\\
38.2 12\\
38.3 12\\
38.4 12\\
38.5 12\\
38.6 12\\
38.7 12\\
38.8 12\\
38.9 12\\
39 12\\
39.1 12\\
39.2 12\\
39.3 12\\
39.4 12\\
39.5 12\\
39.6 12\\
39.7 12\\
39.8 12\\
39.9 12\\
40 12\\
40.1 12\\
40.2 12\\
40.3 12\\
40.4 12\\
40.5 12\\
40.6 12\\
40.7 12\\
40.8 12\\
40.9 12\\
41 12\\
41.1 13\\
41.2 13\\
41.3 13\\
41.4 13\\
41.5 13\\
41.6 13\\
41.7 13\\
41.8 13\\
41.9 13\\
42 13\\
42.1 13\\
42.2 13\\
42.3 13\\
42.4 13\\
42.5 13\\
42.6 13\\
42.7 13\\
42.8 13\\
42.9 13\\
43 13\\
43.1 14\\
43.2 14\\
43.3 14\\
43.4 14\\
43.5 14\\
43.6 14\\
43.7 14\\
43.8 14\\
43.9 14\\
44 14\\
44.1 14\\
44.2 14\\
44.3 14\\
44.4 14\\
44.5 14\\
44.6 14\\
44.7 14\\
44.8 14\\
44.9 14\\
45 14\\
45.1 14\\
45.2 14\\
45.3 14\\
45.4 14\\
45.5 14\\
45.6 14\\
45.7 14\\
45.8 14\\
45.9 14\\
46 14\\
46.1 14\\
46.2 14\\
46.3 14\\
46.4 14\\
46.5 14\\
46.6 14\\
46.7 14\\
46.8 14\\
46.9 14\\
47 14\\
47.1 15\\
47.2 15\\
47.3 15\\
47.4 15\\
47.5 15\\
47.6 15\\
47.7 15\\
47.8 15\\
47.9 15\\
48 15\\
48.1 15\\
48.2 15\\
48.3 15\\
48.4 15\\
48.5 15\\
48.6 15\\
48.7 15\\
48.8 15\\
48.9 15\\
49 15\\
49.1 15\\
49.2 15\\
49.3 15\\
49.4 15\\
49.5 15\\
49.6 15\\
49.7 15\\
49.8 15\\
49.9 15\\
50 15\\
50.1 15\\
50.2 15\\
50.3 15\\
50.4 15\\
50.5 15\\
50.6 15\\
50.7 15\\
50.8 15\\
50.9 15\\
51 15\\
51.1 15\\
51.2 15\\
51.3 15\\
51.4 15\\
51.5 15\\
51.6 15\\
51.7 15\\
51.8 15\\
51.9 15\\
52 15\\
52.1 15\\
52.2 15\\
52.3 15\\
52.4 15\\
52.5 15\\
52.6 15\\
52.7 15\\
52.8 15\\
52.9 15\\
53 15\\
53.1 16\\
53.2 16\\
53.3 16\\
53.4 16\\
53.5 16\\
53.6 16\\
53.7 16\\
53.8 16\\
53.9 16\\
54 16\\
54.1 16\\
54.2 16\\
54.3 16\\
54.4 16\\
54.5 16\\
54.6 16\\
54.7 16\\
54.8 16\\
54.9 16\\
55 16\\
55.1 16\\
55.2 16\\
55.3 16\\
55.4 16\\
55.5 16\\
55.6 16\\
55.7 16\\
55.8 16\\
55.9 16\\
56 16\\
56.1 16\\
56.2 16\\
56.3 16\\
56.4 16\\
56.5 16\\
56.6 16\\
56.7 16\\
56.8 16\\
56.9 16\\
57 16\\
57.1 16\\
57.2 16\\
57.3 16\\
57.4 16\\
57.5 16\\
57.6 16\\
57.7 16\\
57.8 16\\
57.9 16\\
58 16\\
58.1 16\\
58.2 16\\
58.3 16\\
58.4 16\\
58.5 16\\
58.6 16\\
58.7 16\\
58.8 16\\
58.9 16\\
59 16\\
59.1 17\\
59.2 17\\
59.3 17\\
59.4 17\\
59.5 17\\
59.6 17\\
59.7 17\\
59.8 17\\
59.9 17\\
60 17\\
60.1 17\\
60.2 17\\
60.3 17\\
60.4 17\\
60.5 17\\
60.6 17\\
60.7 17\\
60.8 17\\
60.9 17\\
61 17\\
61.1 18\\
61.2 18\\
61.3 18\\
61.4 18\\
61.5 18\\
61.6 18\\
61.7 18\\
61.8 18\\
61.9 18\\
62 18\\
62.1 18\\
62.2 18\\
62.3 18\\
62.4 18\\
62.5 18\\
62.6 18\\
62.7 18\\
62.8 18\\
62.9 18\\
63 18\\
63.1 18\\
63.2 18\\
63.3 18\\
63.4 18\\
63.5 18\\
63.6 18\\
63.7 18\\
63.8 18\\
63.9 18\\
64 18\\
64.1 18\\
64.2 18\\
64.3 18\\
64.4 18\\
64.5 18\\
64.6 18\\
64.7 18\\
64.8 18\\
64.9 18\\
65 18\\
65.1 18\\
65.2 18\\
65.3 18\\
65.4 18\\
65.5 18\\
65.6 18\\
65.7 18\\
65.8 18\\
65.9 18\\
66 18\\
66.1 18\\
66.2 18\\
66.3 18\\
66.4 18\\
66.5 18\\
66.6 18\\
66.7 18\\
66.8 18\\
66.9 18\\
67 18\\
67.1 19\\
67.2 19\\
67.3 19\\
67.4 19\\
67.5 19\\
67.6 19\\
67.7 19\\
67.8 19\\
67.9 19\\
68 19\\
68.1 19\\
68.2 19\\
68.3 19\\
68.4 19\\
68.5 19\\
68.6 19\\
68.7 19\\
68.8 19\\
68.9 19\\
69 19\\
69.1 19\\
69.2 19\\
69.3 19\\
69.4 19\\
69.5 19\\
69.6 19\\
69.7 19\\
69.8 19\\
69.9 19\\
70 19\\
70.1 19\\
70.2 19\\
70.3 19\\
70.4 19\\
70.5 19\\
70.6 19\\
70.7 19\\
70.8 19\\
70.9 19\\
71 19\\
71.1 20\\
71.2 20\\
71.3 20\\
71.4 20\\
71.5 20\\
71.6 20\\
71.7 20\\
71.8 20\\
71.9 20\\
72 20\\
72.1 20\\
72.2 20\\
72.3 20\\
72.4 20\\
72.5 20\\
72.6 20\\
72.7 20\\
72.8 20\\
72.9 20\\
73 20\\
73.1 21\\
73.2 21\\
73.3 21\\
73.4 21\\
73.5 21\\
73.6 21\\
73.7 21\\
73.8 21\\
73.9 21\\
74 21\\
74.1 21\\
74.2 21\\
74.3 21\\
74.4 21\\
74.5 21\\
74.6 21\\
74.7 21\\
74.8 21\\
74.9 21\\
75 21\\
75.1 21\\
75.2 21\\
75.3 21\\
75.4 21\\
75.5 21\\
75.6 21\\
75.7 21\\
75.8 21\\
75.9 21\\
76 21\\
76.1 21\\
76.2 21\\
76.3 21\\
76.4 21\\
76.5 21\\
76.6 21\\
76.7 21\\
76.8 21\\
76.9 21\\
77 21\\
77.1 21\\
77.2 21\\
77.3 21\\
77.4 21\\
77.5 21\\
77.6 21\\
77.7 21\\
77.8 21\\
77.9 21\\
78 21\\
78.1 21\\
78.2 21\\
78.3 21\\
78.4 21\\
78.5 21\\
78.6 21\\
78.7 21\\
78.8 21\\
78.9 21\\
79 21\\
79.1 22\\
79.2 22\\
79.3 22\\
79.4 22\\
79.5 22\\
79.6 22\\
79.7 22\\
79.8 22\\
79.9 22\\
80 22\\
80.1 22\\
80.2 22\\
80.3 22\\
80.4 22\\
80.5 22\\
80.6 22\\
80.7 22\\
80.8 22\\
80.9 22\\
81 22\\
81.1 22\\
81.2 22\\
81.3 22\\
81.4 22\\
81.5 22\\
81.6 22\\
81.7 22\\
81.8 22\\
81.9 22\\
82 22\\
82.1 22\\
82.2 22\\
82.3 22\\
82.4 22\\
82.5 22\\
82.6 22\\
82.7 22\\
82.8 22\\
82.9 22\\
83 22\\
83.1 23\\
83.2 23\\
83.3 23\\
83.4 23\\
83.5 23\\
83.6 23\\
83.7 23\\
83.8 23\\
83.9 23\\
84 23\\
84.1 23\\
84.2 23\\
84.3 23\\
84.4 23\\
84.5 23\\
84.6 23\\
84.7 23\\
84.8 23\\
84.9 23\\
85 23\\
85.1 23\\
85.2 23\\
85.3 23\\
85.4 23\\
85.5 23\\
85.6 23\\
85.7 23\\
85.8 23\\
85.9 23\\
86 23\\
86.1 23\\
86.2 23\\
86.3 23\\
86.4 23\\
86.5 23\\
86.6 23\\
86.7 23\\
86.8 23\\
86.9 23\\
87 23\\
87.1 23\\
87.2 23\\
87.3 23\\
87.4 23\\
87.5 23\\
87.6 23\\
87.7 23\\
87.8 23\\
87.9 23\\
88 23\\
88.1 23\\
88.2 23\\
88.3 23\\
88.4 23\\
88.5 23\\
88.6 23\\
88.7 23\\
88.8 23\\
88.9 23\\
89 23\\
89.1 24\\
89.2 24\\
89.3 24\\
89.4 24\\
89.5 24\\
89.6 24\\
89.7 24\\
89.8 24\\
89.9 24\\
90 24\\
90.1 24\\
90.2 24\\
90.3 24\\
90.4 24\\
90.5 24\\
90.6 24\\
90.7 24\\
90.8 24\\
90.9 24\\
91 24\\
91.1 24\\
91.2 24\\
91.3 24\\
91.4 24\\
91.5 24\\
91.6 24\\
91.7 24\\
91.8 24\\
91.9 24\\
92 24\\
92.1 24\\
92.2 24\\
92.3 24\\
92.4 24\\
92.5 24\\
92.6 24\\
92.7 24\\
92.8 24\\
92.9 24\\
93 24\\
93.1 24\\
93.2 24\\
93.3 24\\
93.4 24\\
93.5 24\\
93.6 24\\
93.7 24\\
93.8 24\\
93.9 24\\
94 24\\
94.1 24\\
94.2 24\\
94.3 24\\
94.4 24\\
94.5 24\\
94.6 24\\
94.7 24\\
94.8 24\\
94.9 24\\
95 24\\
95.1 24\\
95.2 24\\
95.3 24\\
95.4 24\\
95.5 24\\
95.6 24\\
95.7 24\\
95.8 24\\
95.9 24\\
96 24\\
96.1 24\\
96.2 24\\
96.3 24\\
96.4 24\\
96.5 24\\
96.6 24\\
96.7 24\\
96.8 24\\
96.9 24\\
97 24\\
97.1 25\\
97.2 25\\
97.3 25\\
97.4 25\\
97.5 25\\
97.6 25\\
97.7 25\\
97.8 25\\
97.9 25\\
98 25\\
98.1 25\\
98.2 25\\
98.3 25\\
98.4 25\\
98.5 25\\
98.6 25\\
98.7 25\\
98.8 25\\
98.9 25\\
99 25\\
99.1 25\\
99.2 25\\
99.3 25\\
99.4 25\\
99.5 25\\
99.6 25\\
99.7 25\\
99.8 25\\
99.9 25\\
100 25\\
};
\end{axis}
\end{tikzpicture}%
	\caption{$ \pi(x) $ von $ [0,100] $}
	\label{fig:pi}
\end{figure}

\begin{df}[Landau-Symbolik, asymptotische Gleichheit]
Sei $ f,g: \R\to \R $, dann sagen wir:
\begin{itemize}
\item $ f=\mathcal O(g)$,wenn es ein $ c\in \R_+ $ gibt, sodass $ \forall_{x\in \R} |f(x)|\le c|g(x)| $.
\item $ f\sim g $, wenn $ \lim_{x\to +\infty} \frac{f(x)}{g(x)}=1 $.
\end{itemize}
\end{df}

Nun widmen wir uns dem Primzahlensatz.
\begin{st}[Primzahlensatz]
$ \pi(x)\sim \frac{x}{\ln x} $
\end{st}
\begin{figure}[H]
	\centering
	\setlength\figureheight{3cm} 
	\setlength\figurewidth{5cm}
	\begin{minipage}{5cm}% This file was created by matlab2tikz v0.3.3.
% Copyright (c) 2008--2013, Nico Schlömer <nico.schloemer@gmail.com>
% All rights reserved.
% 
% The latest updates can be retrieved from
%   http://www.mathworks.com/matlabcentral/fileexchange/22022-matlab2tikz
% where you can also make suggestions and rate matlab2tikz.
% 
% 
% 
\begin{tikzpicture}

\begin{axis}[%
width=\figurewidth,
height=\figureheight,
unbounded coords=jump,
scale only axis,
xmin=0,
xmax=50000000,
ymin=1.06,
ymax=1.105
]
\addplot [
color=red,
dashed,
forget plot
]
table[row sep=crcr]{
100000 1.10431981059994\\
200000 1.09757005228607\\
300000 1.09287382327112\\
400000 1.09191895827853\\
500000 1.09015345994124\\
600000 1.08872236816545\\
700000 1.08714706017692\\
800000 1.08655682805285\\
900000 1.08575248235039\\
1000000 1.08448994777908\\
1100000 1.08395644428827\\
1200000 1.08410876756783\\
1300000 1.08314085970342\\
1400000 1.0828895063234\\
1500000 1.08226365144033\\
1600000 1.08147592309637\\
1700000 1.0813697488985\\
1800000 1.08082342360389\\
1900000 1.08071842875594\\
2000000 1.08040896148581\\
2100000 1.08005865260914\\
2200000 1.0797775588778\\
2300000 1.07959489936041\\
2400000 1.07918709655133\\
2500000 1.07879213029332\\
2600000 1.0787390992031\\
2700000 1.07854409947203\\
2800000 1.07819118643054\\
2900000 1.07809255726278\\
3000000 1.07787348637182\\
3100000 1.07758561494255\\
3200000 1.07756957962905\\
3300000 1.07749535877662\\
3400000 1.07725078547585\\
3500000 1.07695103504111\\
3600000 1.07656938497612\\
3700000 1.07664188490095\\
3800000 1.07643189000446\\
3900000 1.07640596827909\\
4000000 1.07608256390475\\
4100000 1.07615685262036\\
4200000 1.07594400749981\\
4300000 1.07566786159045\\
4400000 1.07543934070788\\
4500000 1.07559848336097\\
4600000 1.07538046964691\\
4700000 1.07529744006541\\
4800000 1.0750908335296\\
4900000 1.0751632271444\\
5000000 1.07515901325279\\
5100000 1.07498798860779\\
5200000 1.07478057540766\\
5300000 1.07476899501101\\
5400000 1.07468997195501\\
5500000 1.07456627163585\\
5600000 1.07436642700776\\
5700000 1.0741975048934\\
5800000 1.07400661665988\\
5900000 1.07396885586297\\
6000000 1.07390763724272\\
6100000 1.07380579894199\\
6200000 1.07373551532892\\
6300000 1.07373982461306\\
6400000 1.07354337276145\\
6500000 1.07339110000319\\
6600000 1.07338793691036\\
6700000 1.07324822697627\\
6800000 1.07321106463266\\
6900000 1.07326977329625\\
7000000 1.0732356653085\\
7100000 1.07321887524954\\
7200000 1.07307599503953\\
7300000 1.07304117573646\\
7400000 1.0729092181841\\
7500000 1.07279757811004\\
7600000 1.07270951225353\\
7700000 1.07271814107214\\
7800000 1.07253092248534\\
7900000 1.07257486823958\\
8000000 1.0724661949362\\
8100000 1.07232201473034\\
8200000 1.07228320095325\\
8300000 1.07222129312643\\
8400000 1.07217301175669\\
8500000 1.07206270842163\\
8600000 1.07210767627538\\
8700000 1.07207688119908\\
8800000 1.07208530527821\\
8900000 1.07196440451797\\
9000000 1.07194408654801\\
9100000 1.0717834336445\\
9200000 1.07174111462787\\
9300000 1.071732371711\\
9400000 1.07166729607644\\
9500000 1.07153394773383\\
9600000 1.07156871107766\\
9700000 1.07149297678886\\
9800000 1.07143797355958\\
9900000 1.07133958560648\\
10000000 1.07117478896182\\
10100000 1.07118968371473\\
10200000 1.07106598014019\\
10300000 1.07104099747846\\
10400000 1.07101567978115\\
10500000 1.07103154999764\\
10600000 1.07098978922086\\
10700000 1.07099320974561\\
10800000 1.07090412193389\\
10900000 1.07088096029311\\
11000000 1.07084681490438\\
11100000 1.07083845260134\\
11200000 1.07080010657226\\
11300000 1.07071235722627\\
11400000 1.0706333942254\\
11500000 1.07059830096872\\
11600000 1.07058213019051\\
11700000 1.07059412196052\\
11800000 1.07054110592474\\
11900000 1.07049166660654\\
12000000 1.0704755653951\\
12100000 1.07037607529552\\
12200000 1.07034488012443\\
12300000 1.0703365803226\\
12400000 1.07035325534294\\
12500000 1.07028972468945\\
12600000 1.0702256023467\\
12700000 1.07021882704601\\
12800000 1.0702412559139\\
12900000 1.07018306895486\\
13000000 1.0700875649304\\
13100000 1.0699580188084\\
13200000 1.06994666419459\\
13300000 1.06989063538732\\
13400000 1.06989126865871\\
13500000 1.06978560460089\\
13600000 1.06971350481322\\
13700000 1.0697378251989\\
13800000 1.06972910041356\\
13900000 1.06974004011863\\
14000000 1.06963741281605\\
14100000 1.06959742221605\\
14200000 1.06953297852449\\
14300000 1.06953440902178\\
14400000 1.06959353429732\\
14500000 1.06950223792322\\
14600000 1.06950746186148\\
14700000 1.06942433781784\\
14800000 1.06942297657942\\
14900000 1.06937876597607\\
15000000 1.06929910153792\\
15100000 1.06927548746867\\
15200000 1.06920894996821\\
15300000 1.06921055084759\\
15400000 1.06912104848559\\
15500000 1.06907239716717\\
15600000 1.06911267999713\\
15700000 1.06909358769987\\
15800000 1.06902875276615\\
15900000 1.06905037340878\\
16000000 1.0690304256748\\
16100000 1.06896240738589\\
16200000 1.06890247509721\\
16300000 1.06891160255304\\
16400000 1.06882955030312\\
16500000 1.06884318803563\\
16600000 1.06878662953322\\
16700000 1.06875864111696\\
16800000 1.06878744780186\\
16900000 1.06872831014556\\
17000000 1.06876385155351\\
17100000 1.06877659699436\\
17200000 1.06864776651932\\
17300000 1.06864939823486\\
17400000 1.0686576782674\\
17500000 1.06869726587141\\
17600000 1.06869276187953\\
17700000 1.06863260012369\\
17800000 1.06859643529691\\
17900000 1.06854469273833\\
18000000 1.06858897802029\\
18100000 1.06849234637191\\
18200000 1.06841862988582\\
18300000 1.06846798170075\\
18400000 1.06841395102349\\
18500000 1.06835940875628\\
18600000 1.0683484490373\\
18700000 1.06832488939973\\
18800000 1.06830850964947\\
18900000 1.06827880381731\\
19000000 1.06827036247064\\
19100000 1.06822229688036\\
19200000 1.06819981634736\\
19300000 1.06812952476874\\
19400000 1.06815137196707\\
19500000 1.06812972844752\\
19600000 1.06804550824303\\
19700000 1.06804027756619\\
19800000 1.06805095840433\\
19900000 1.06801228795291\\
20000000 1.06802414102135\\
20100000 1.06802054332889\\
20200000 1.06799004486398\\
20300000 1.0679280388004\\
20400000 1.06791003313182\\
20500000 1.06784425412947\\
20600000 1.06785888130386\\
20700000 1.06785027232499\\
20800000 1.06783736826059\\
20900000 1.06780570369229\\
21000000 1.06775632678018\\
21100000 1.06774142077138\\
21200000 1.06770879739119\\
21300000 1.06770938673854\\
21400000 1.06767806805176\\
21500000 1.06769378212485\\
21600000 1.0676401925163\\
21700000 1.06762017436109\\
21800000 1.06762631071687\\
21900000 1.06759728774309\\
22000000 1.06761885195741\\
22100000 1.06757786576015\\
22200000 1.067517026921\\
22300000 1.06746389593748\\
22400000 1.06742516995389\\
22500000 1.06738636694261\\
22600000 1.06741191938596\\
22700000 1.06740775118004\\
22800000 1.06736159275312\\
22900000 1.06736868019713\\
23000000 1.0673391775936\\
23100000 1.06733811889968\\
23200000 1.06730301616738\\
23300000 1.06729907497506\\
23400000 1.0672751546304\\
23500000 1.06726832432671\\
23600000 1.06724024900768\\
23700000 1.06722915144154\\
23800000 1.06717701873604\\
23900000 1.06714688619051\\
24000000 1.06714061450175\\
24100000 1.06711278037727\\
24200000 1.06711140484613\\
24300000 1.06712987750561\\
24400000 1.06712474198374\\
24500000 1.06708029371846\\
24600000 1.06707169412013\\
24700000 1.06707578670953\\
24800000 1.06706152826153\\
24900000 1.06702707104523\\
25000000 1.06698422261239\\
25100000 1.0669771966577\\
25200000 1.06693863560985\\
25300000 1.06686956142167\\
25400000 1.06686562985367\\
25500000 1.06684117892588\\
25600000 1.06683041426061\\
25700000 1.06679069270201\\
25800000 1.06679971943176\\
25900000 1.06680555218176\\
26000000 1.0667898351854\\
26100000 1.06677046451429\\
26200000 1.0667331330297\\
26300000 1.06671440906657\\
26400000 1.0667024385765\\
26500000 1.06671520636871\\
26600000 1.06672031028866\\
26700000 1.06669028859302\\
26800000 1.06665231296664\\
26900000 1.06665161768462\\
27000000 1.06664851674525\\
27100000 1.06666513642174\\
27200000 1.0666238569244\\
27300000 1.06657984029066\\
27400000 1.06656873884794\\
27500000 1.06652854448484\\
27600000 1.06657067035686\\
27700000 1.0665198111277\\
27800000 1.06651440331262\\
27900000 1.06649437864369\\
28000000 1.06648376611108\\
28100000 1.06649040301882\\
28200000 1.06643320623172\\
28300000 1.06646499238276\\
28400000 1.0664350089224\\
28500000 1.06643180074563\\
28600000 1.06641187818992\\
28700000 1.06640053355601\\
28800000 1.0663750174457\\
28900000 1.06634616579188\\
29000000 1.06632704376006\\
29100000 1.06629865535799\\
29200000 1.06627345019622\\
29300000 1.06626665175558\\
29400000 1.06626755218388\\
29500000 1.06622068581804\\
29600000 1.06621426631733\\
29700000 1.06620966939468\\
29800000 1.06616356163099\\
29900000 1.06619607823608\\
30000000 1.06620719320021\\
30100000 1.06622284950686\\
30200000 1.066214486968\\
30300000 1.06621473401764\\
30400000 1.06621614034981\\
30500000 1.06615710542916\\
30600000 1.0661767533795\\
30700000 1.06618843954737\\
30800000 1.06615416591179\\
30900000 1.06616979129838\\
31000000 1.06611856365807\\
31100000 1.06607043226448\\
31200000 1.06604472458619\\
31300000 1.06603517055628\\
31400000 1.06602458033744\\
31500000 1.06598445948253\\
31600000 1.06597846149257\\
31700000 1.06597140956375\\
31800000 1.06593343305114\\
31900000 1.06594334097462\\
32000000 1.06593698384228\\
32100000 1.06590373397527\\
32200000 1.06590288221489\\
32300000 1.06589292414532\\
32400000 1.06588674754585\\
32500000 1.06587846345013\\
32600000 1.06588188576875\\
32700000 1.06586675055341\\
32800000 1.0658469341433\\
32900000 1.06582510557334\\
33000000 1.06582593686817\\
33100000 1.06578279742193\\
33200000 1.06578682331427\\
33300000 1.06578611735575\\
33400000 1.06579420377013\\
33500000 1.06578307655846\\
33600000 1.06575805354104\\
33700000 1.06574033732385\\
33800000 1.06573242967379\\
33900000 1.06574447665913\\
34000000 1.06569827237208\\
34100000 1.06565889622734\\
34200000 1.06561310185979\\
34300000 1.0655953384191\\
34400000 1.06553676684073\\
34500000 1.06553682801592\\
34600000 1.06553784568209\\
34700000 1.06548827423038\\
34800000 1.06547634688831\\
34900000 1.06547587936684\\
35000000 1.06546642893651\\
35100000 1.06542430763556\\
35200000 1.0654164195467\\
35300000 1.06540999497485\\
35400000 1.06540993108618\\
35500000 1.06539364875564\\
35600000 1.06538032395608\\
35700000 1.06538162311136\\
35800000 1.06537653779097\\
35900000 1.06534088821916\\
36000000 1.06533387655107\\
36100000 1.06533069286396\\
36200000 1.06529957388223\\
36300000 1.06532033972087\\
36400000 1.06530505730356\\
36500000 1.06530027987533\\
36600000 1.06528879495904\\
36700000 1.065255465404\\
36800000 1.06525016453702\\
36900000 1.06520657004203\\
37000000 1.06518949829684\\
37100000 1.06519451780032\\
37200000 1.06521021307773\\
37300000 1.06516732490563\\
37400000 1.0651753940916\\
37500000 1.06515636966069\\
37600000 1.06515359666942\\
37700000 1.06514566616876\\
37800000 1.06513030771508\\
37900000 1.06510158997615\\
38000000 1.06508257586937\\
38100000 1.06508693690469\\
38200000 1.06510033114098\\
38300000 1.06511813180345\\
38400000 1.06509801080627\\
38500000 1.06508561237859\\
38600000 1.06507318536724\\
38700000 1.06504899107092\\
38800000 1.06507796593082\\
38900000 1.06507660780725\\
39000000 1.06505902933213\\
39100000 1.06504725476745\\
39200000 1.06503142866204\\
39300000 1.06505785450319\\
39400000 1.06505209827634\\
39500000 1.06505247293101\\
39600000 1.06504612218722\\
39700000 1.06507143625284\\
39800000 1.06504596841923\\
39900000 1.06501876725077\\
40000000 1.06499071926135\\
40100000 1.0649823487606\\
40200000 1.06496999626803\\
40300000 1.06497280846021\\
40400000 1.06498330833211\\
40500000 1.06496727268628\\
40600000 1.0649227279837\\
40700000 1.06490971230656\\
40800000 1.0649421800931\\
40900000 1.06491482741477\\
41000000 1.06489433468022\\
41100000 1.06489089116398\\
41200000 1.064891611479\\
41300000 1.06485697756254\\
41400000 1.06483933817202\\
41500000 1.06483392710606\\
41600000 1.06480354886816\\
41700000 1.06478413727208\\
41800000 1.06477939560813\\
41900000 1.06478377863321\\
42000000 1.06473411620505\\
42100000 1.06471959101523\\
42200000 1.06471791521546\\
42300000 1.06474643781222\\
42400000 1.06473205490072\\
42500000 1.0647291929374\\
42600000 1.06469818618816\\
42700000 1.06469559157643\\
42800000 1.06468755440966\\
42900000 1.0646511689054\\
43000000 1.06463158452129\\
43100000 1.06462298005398\\
43200000 1.0646106317028\\
43300000 1.06458969011891\\
43400000 1.06459181670803\\
43500000 1.06459462262778\\
43600000 1.06459648893622\\
43700000 1.06454951367668\\
43800000 1.06453917538972\\
43900000 1.06452114340335\\
44000000 1.06453546580621\\
44100000 1.0645144800187\\
44200000 1.06449943485352\\
44300000 1.06450897123763\\
44400000 1.06447789047972\\
44500000 1.06446304285098\\
44600000 1.06447577721472\\
44700000 1.06447926861077\\
44800000 1.06447908030729\\
44900000 1.06445443623703\\
45000000 1.06444190869845\\
45100000 1.06445314453668\\
45200000 1.06442403936614\\
45300000 1.06444396155354\\
45400000 1.06445745933865\\
45500000 1.06445720873815\\
45600000 1.06445992434925\\
45700000 1.06445323699278\\
45800000 1.06443104177997\\
45900000 1.06444032523442\\
46000000 1.06442297421004\\
46100000 1.06441819554517\\
46200000 1.06439305943803\\
46300000 1.06436636968121\\
46400000 1.06436248384116\\
46500000 1.06435658352996\\
46600000 1.06434337568013\\
46700000 1.06435693681591\\
46800000 1.06434615813011\\
46900000 1.06434621170073\\
47000000 1.06433447981155\\
47100000 1.0642949028915\\
47200000 1.06431038142813\\
47300000 1.06429726909193\\
47400000 1.06428929454102\\
47500000 1.06428196102029\\
47600000 1.06426449374225\\
47700000 1.06423917624224\\
47800000 1.06421641242106\\
47900000 1.0642323681208\\
48000000 1.06420353952852\\
48100000 1.06418241021688\\
48200000 1.06417113660518\\
48300000 1.06414987946485\\
48400000 1.0641552565912\\
48500000 1.06413712232906\\
48600000 1.0641247474814\\
48700000 1.06411628067583\\
48800000 1.0640837653325\\
48900000 1.06408997849012\\
49000000 1.06407723740753\\
49100000 1.06407270183583\\
49200000 1.0640644444575\\
49300000 1.06406039072059\\
49400000 1.06407019857907\\
49500000 1.06407338631895\\
49600000 1.0640832094202\\
49700000 1.06407003432011\\
49800000 1.06406103922881\\
49900000 1.06406792224301\\
50000000 1.06405407426476\\
};
\end{axis}
\end{tikzpicture}%\end{minipage} \hspace{3cm} \begin{minipage}{5cm}% This file was created by matlab2tikz v0.3.3.
% Copyright (c) 2008--2013, Nico Schlömer <nico.schloemer@gmail.com>
% All rights reserved.
% 
% The latest updates can be retrieved from
%   http://www.mathworks.com/matlabcentral/fileexchange/22022-matlab2tikz
% where you can also make suggestions and rate matlab2tikz.
% 
% 
% 
\begin{tikzpicture}

\begin{axis}[%
width=\figurewidth,
height=\figureheight,
scale only axis,
xmin=0,
xmax=50000000,
ymin=-0,
ymax=3500000,
axis x line*=bottom,
axis y line*=left
]
\addplot [
color=blue,
solid,
forget plot
]
table[row sep=crcr]{
0 0\\
100000 9592\\
200000 17984\\
300000 25997\\
400000 33860\\
500000 41538\\
600000 49098\\
700000 56543\\
800000 63951\\
900000 71274\\
1000000 78498\\
1100000 85714\\
1200000 92938\\
1300000 100021\\
1400000 107126\\
1500000 114155\\
1600000 121127\\
1700000 128141\\
1800000 135072\\
1900000 142029\\
2000000 148933\\
2100000 155805\\
2200000 162662\\
2300000 169511\\
2400000 176302\\
2500000 183072\\
2600000 189880\\
2700000 196645\\
2800000 203362\\
2900000 210109\\
3000000 216816\\
3100000 223492\\
3200000 230209\\
3300000 236900\\
3400000 243539\\
3500000 250150\\
3600000 256726\\
3700000 263397\\
3800000 269987\\
3900000 276611\\
4000000 283146\\
4100000 289774\\
4200000 296314\\
4300000 302824\\
4400000 309335\\
4500000 315948\\
4600000 322441\\
4700000 328964\\
4800000 335439\\
4900000 341992\\
5000000 348513\\
5100000 354971\\
5200000 361407\\
5300000 367900\\
5400000 374362\\
5500000 380800\\
5600000 387202\\
5700000 393606\\
5800000 399993\\
5900000 406429\\
6000000 412849\\
6100000 419246\\
6200000 425648\\
6300000 432073\\
6400000 438410\\
6500000 444757\\
6600000 451159\\
6700000 457497\\
6800000 463872\\
6900000 470283\\
7000000 476648\\
7100000 483015\\
7200000 489319\\
7300000 495666\\
7400000 501962\\
7500000 508261\\
7600000 514565\\
7700000 520910\\
7800000 527154\\
7900000 533506\\
8000000 539777\\
8100000 546024\\
8200000 552319\\
8300000 558597\\
8400000 564877\\
8500000 571119\\
8600000 577439\\
8700000 583714\\
8800000 590006\\
8900000 596222\\
9000000 602489\\
9100000 608672\\
9200000 614917\\
9300000 621177\\
9400000 627400\\
9500000 633578\\
9600000 639851\\
9700000 646054\\
9800000 652265\\
9900000 658445\\
10000000 664579\\
10100000 670820\\
10200000 676970\\
10300000 683178\\
10400000 689382\\
10500000 695609\\
10600000 701795\\
10700000 708007\\
10800000 714154\\
10900000 720341\\
11000000 726517\\
11100000 732707\\
11200000 738873\\
11300000 745001\\
11400000 751131\\
11500000 757288\\
11600000 763455\\
11700000 769639\\
11800000 775773\\
11900000 781906\\
12000000 788060\\
12100000 794149\\
12200000 800285\\
12300000 806435\\
12400000 812601\\
12500000 818703\\
12600000 824801\\
12700000 830940\\
12800000 837099\\
12900000 843192\\
13000000 849252\\
13100000 855281\\
13200000 861401\\
13300000 867482\\
13400000 873606\\
13500000 879640\\
13600000 885698\\
13700000 891833\\
13800000 897938\\
13900000 904057\\
14000000 910077\\
14100000 916147\\
14200000 922193\\
14300000 928293\\
14400000 934441\\
14500000 940455\\
14600000 946551\\
14700000 952566\\
14800000 958651\\
14900000 964695\\
15000000 970704\\
15100000 976761\\
15200000 982776\\
15300000 988851\\
15400000 994839\\
15500000 1000862\\
15600000 1006966\\
15700000 1013012\\
15800000 1019012\\
15900000 1025092\\
16000000 1031130\\
16100000 1037119\\
16200000 1043113\\
16300000 1049172\\
16400000 1055139\\
16500000 1061198\\
16600000 1067185\\
16700000 1073198\\
16800000 1079266\\
16900000 1085243\\
17000000 1091314\\
17100000 1097360\\
17200000 1103258\\
17300000 1109288\\
17400000 1115323\\
17500000 1121389\\
17600000 1127407\\
17700000 1133364\\
17800000 1139344\\
17900000 1145305\\
18000000 1151367\\
18100000 1157275\\
18200000 1163205\\
18300000 1169267\\
18400000 1175214\\
18500000 1181158\\
18600000 1187148\\
18700000 1193122\\
18800000 1199102\\
18900000 1205065\\
19000000 1211050\\
19100000 1216988\\
19200000 1222953\\
19300000 1228861\\
19400000 1234873\\
19500000 1240833\\
19600000 1246718\\
19700000 1252693\\
19800000 1258685\\
19900000 1264617\\
20000000 1270607\\
20100000 1276577\\
20200000 1282513\\
20300000 1288409\\
20400000 1294356\\
20500000 1300243\\
20600000 1306226\\
20700000 1312179\\
20800000 1318125\\
20900000 1324046\\
21000000 1329943\\
21100000 1335881\\
21200000 1341795\\
21300000 1347749\\
21400000 1353661\\
21500000 1359631\\
21600000 1365511\\
21700000 1371432\\
21800000 1377385\\
21900000 1383291\\
22000000 1389261\\
22100000 1395148\\
22200000 1401007\\
22300000 1406874\\
22400000 1412758\\
22500000 1418640\\
22600000 1424606\\
22700000 1430531\\
22800000 1436398\\
22900000 1442335\\
23000000 1448221\\
23100000 1454144\\
23200000 1460019\\
23300000 1465935\\
23400000 1471822\\
23500000 1477731\\
23600000 1483609\\
23700000 1489509\\
23800000 1495350\\
23900000 1501220\\
24000000 1507122\\
24100000 1512992\\
24200000 1518898\\
24300000 1524831\\
24400000 1530729\\
24500000 1536569\\
24600000 1542459\\
24700000 1548366\\
24800000 1554245\\
24900000 1560093\\
25000000 1565927\\
25100000 1571812\\
25200000 1577649\\
25300000 1583439\\
25400000 1589324\\
25500000 1595177\\
25600000 1601049\\
25700000 1606876\\
25800000 1612775\\
25900000 1618668\\
26000000 1624527\\
26100000 1630379\\
26200000 1636202\\
26300000 1642052\\
26400000 1647911\\
26500000 1653807\\
26600000 1659690\\
26700000 1665517\\
26800000 1671330\\
26900000 1677200\\
27000000 1683065\\
27100000 1688960\\
27200000 1694762\\
27300000 1700558\\
27400000 1706405\\
27500000 1712204\\
27600000 1718134\\
27700000 1723913\\
27800000 1729764\\
27900000 1735590\\
28000000 1741430\\
28100000 1747297\\
28200000 1753058\\
28300000 1758964\\
28400000 1764767\\
28500000 1770613\\
28600000 1776430\\
28700000 1782260\\
28800000 1788065\\
28900000 1793863\\
29000000 1799676\\
29100000 1805472\\
29200000 1811272\\
29300000 1817102\\
29400000 1822944\\
29500000 1828703\\
29600000 1834530\\
29700000 1840359\\
29800000 1846115\\
29900000 1852006\\
30000000 1857859\\
30100000 1863719\\
30200000 1869536\\
30300000 1875367\\
30400000 1881199\\
30500000 1886923\\
30600000 1892785\\
30700000 1898632\\
30800000 1904396\\
30900000 1910248\\
31000000 1915979\\
31100000 1921714\\
31200000 1927488\\
31300000 1933290\\
31400000 1939089\\
31500000 1944833\\
31600000 1950638\\
31700000 1956440\\
31800000 1962184\\
31900000 1968015\\
32000000 1973815\\
32100000 1979564\\
32200000 1985372\\
32300000 1991162\\
32400000 1996958\\
32500000 2002749\\
32600000 2008561\\
32700000 2014337\\
32800000 2020103\\
32900000 2025864\\
33000000 2031667\\
33100000 2037385\\
33200000 2043192\\
33300000 2048989\\
33400000 2054802\\
33500000 2060577\\
33600000 2066324\\
33700000 2072084\\
33800000 2077862\\
33900000 2083678\\
34000000 2089379\\
34100000 2095092\\
34200000 2100791\\
34300000 2106544\\
34400000 2112215\\
34500000 2118001\\
34600000 2123788\\
34700000 2129473\\
34800000 2135232\\
34900000 2141013\\
35000000 2146775\\
35100000 2152470\\
35200000 2158233\\
35300000 2163998\\
35400000 2169775\\
35500000 2175518\\
35600000 2181266\\
35700000 2187043\\
35800000 2192806\\
35900000 2198505\\
36000000 2204262\\
36100000 2210026\\
36200000 2215731\\
36300000 2221543\\
36400000 2227279\\
36500000 2233036\\
36600000 2238778\\
36700000 2244473\\
36800000 2250226\\
36900000 2255897\\
37000000 2261623\\
37100000 2267395\\
37200000 2273189\\
37300000 2278857\\
37400000 2284633\\
37500000 2290350\\
37600000 2296101\\
37700000 2301840\\
37800000 2307562\\
37900000 2313254\\
38000000 2318966\\
38100000 2324728\\
38200000 2330509\\
38300000 2336299\\
38400000 2342005\\
38500000 2347727\\
38600000 2353448\\
38700000 2359142\\
38800000 2364953\\
38900000 2370696\\
39000000 2376402\\
39100000 2382120\\
39200000 2387828\\
39300000 2393630\\
39400000 2399359\\
39500000 2405101\\
39600000 2410827\\
39700000 2416624\\
39800000 2422305\\
39900000 2427981\\
40000000 2433654\\
40100000 2439371\\
40200000 2445078\\
40300000 2450819\\
40400000 2456577\\
40500000 2462273\\
40600000 2467902\\
40700000 2473603\\
40800000 2479409\\
40900000 2485075\\
41000000 2490756\\
41100000 2496476\\
41200000 2502205\\
41300000 2507850\\
41400000 2513534\\
41500000 2519246\\
41600000 2524898\\
41700000 2530575\\
41800000 2536286\\
41900000 2542018\\
42000000 2547620\\
42100000 2553305\\
42200000 2559020\\
42300000 2564807\\
42400000 2570490\\
42500000 2576200\\
42600000 2581841\\
42700000 2587550\\
42800000 2593245\\
42900000 2598870\\
43000000 2604535\\
43100000 2610226\\
43200000 2615907\\
43300000 2621566\\
43400000 2627281\\
43500000 2632997\\
43600000 2638710\\
43700000 2644301\\
43800000 2649982\\
43900000 2655643\\
44000000 2661384\\
44100000 2667036\\
44200000 2672702\\
44300000 2678429\\
44400000 2684053\\
44500000 2689717\\
44600000 2695450\\
44700000 2701159\\
44800000 2706858\\
44900000 2712494\\
45000000 2718160\\
45100000 2723886\\
45200000 2729508\\
45300000 2735255\\
45400000 2740985\\
45500000 2746679\\
45600000 2752380\\
45700000 2758056\\
45800000 2763691\\
45900000 2769407\\
46000000 2775053\\
46100000 2780731\\
46200000 2786355\\
46300000 2791974\\
46400000 2797652\\
46500000 2803324\\
46600000 2808976\\
46700000 2814698\\
46800000 2820355\\
46900000 2826040\\
47000000 2831693\\
47100000 2837271\\
47200000 2842995\\
47300000 2848642\\
47400000 2854302\\
47500000 2859963\\
47600000 2865596\\
47700000 2871207\\
47800000 2876824\\
47900000 2882545\\
48000000 2888144\\
48100000 2893763\\
48200000 2899408\\
48300000 2905025\\
48400000 2910714\\
48500000 2916338\\
48600000 2921977\\
48700000 2927626\\
48800000 2933208\\
48900000 2938896\\
49000000 2944531\\
49100000 2950188\\
49200000 2955834\\
49300000 2961491\\
49400000 2967186\\
49500000 2972862\\
49600000 2978556\\
49700000 2984185\\
49800000 2989825\\
49900000 2995509\\
50000000 3001134\\
};
\addplot [
color=black,
dotted,
forget plot
]
table[row sep=crcr]{
0 -0\\
100000 8685.88963806504\\
200000 16385.2867181844\\
300000 23787.7414999176\\
400000 31009.6273567612\\
500000 38102.8924150173\\
600000 45096.8965418913\\
700000 52010.4428105597\\
800000 58856.5626287607\\
900000 65644.7958062313\\
1000000 72382.413650542\\
1100000 79075.1330015664\\
1200000 85727.5605366643\\
1300000 92343.483401953\\
1400000 98926.0671328428\\
1500000 105477.994985858\\
1600000 112001.568794247\\
1700000 118498.78372363\\
1800000 124971.384825855\\
1900000 131420.910591388\\
2000000 137848.727018316\\
2100000 144256.054635196\\
2200000 150643.990202068\\
2300000 157013.52433253\\
2400000 163365.555947986\\
2500000 169700.904242065\\
2600000 176020.318666738\\
2700000 182324.487330895\\
2800000 188614.044113317\\
2900000 194889.574725806\\
3000000 201151.621912341\\
3100000 207400.689932107\\
3200000 213637.248445014\\
3300000 219861.735895525\\
3400000 226074.562472863\\
3500000 232276.112711523\\
3600000 238466.747784858\\
3700000 244646.807535481\\
3800000 250816.612278999\\
3900000 256976.464411688\\
4000000 263126.649847904\\
4100000 269267.439309078\\
4200000 275399.089482872\\
4300000 281521.844068348\\
4400000 287635.93472077\\
4500000 293741.581907724\\
4600000 299838.995686679\\
4700000 305928.376412753\\
4800000 312009.915384293\\
4900000 318083.795432923\\
5000000 324150.191463872\\
5100000 330209.270951688\\
5200000 336261.194395813\\
5300000 342306.115739997\\
5400000 348344.182759035\\
5500000 354375.537415942\\
5600000 360400.316192312\\
5700000 366418.650394335\\
5800000 372430.666436638\\
5900000 378436.48610594\\
6000000 384436.226806244\\
6100000 390430.001787177\\
6200000 396417.920356867\\
6300000 402400.088080654\\
6400000 408376.606966784\\
6500000 414347.575640117\\
6600000 420313.08950482\\
6700000 426273.240896876\\
6800000 432228.119227202\\
6900000 438177.811116079\\
7000000 444122.400519545\\
7100000 450061.968848333\\
7200000 455996.595079901\\
7300000 461926.355864031\\
7400000 467851.325622471\\
7500000 473771.576643013\\
7600000 479687.179168394\\
7700000 485598.201480375\\
7800000 491504.709979312\\
7900000 497406.769259516\\
8000000 503304.44218068\\
8100000 509197.789935619\\
8200000 515086.872114563\\
8300000 520971.746766212\\
8400000 526852.470455755\\
8500000 532729.098320047\\
8600000 538601.684120093\\
8700000 544470.280291035\\
8800000 550334.937989744\\
8900000 556195.707140205\\
9000000 562052.636476779\\
9100000 567905.773585495\\
9200000 573755.164943461\\
9300000 579600.855956515\\
9400000 585442.890995201\\
9500000 591281.313429167\\
9600000 597116.165660073\\
9700000 602947.48915308\\
9800000 608775.324467003\\
9900000 614599.711283192\\
10000000 620420.688433217\\
10100000 626238.293925398\\
10200000 632052.564970266\\
10300000 637863.538004984\\
10400000 643671.248716794\\
10500000 649475.732065534\\
10600000 655277.022305273\\
10700000 661075.153005097\\
10800000 666870.1570691\\
10900000 672662.066755611\\
11000000 678450.913695693\\
11100000 684236.728910945\\
11200000 690019.542830645\\
11300000 695799.385308266\\
11400000 701576.285637383\\
11500000 707350.272567006\\
11600000 713121.374316365\\
11700000 718889.618589168\\
11800000 724655.032587358\\
11900000 730417.643024387\\
12000000 736177.476138031\\
12100000 741934.557702765\\
12200000 747688.913041715\\
12300000 753440.567038209\\
12400000 759189.544146938\\
12500000 764935.868404745\\
12600000 770679.563441061\\
12700000 776420.652487999\\
12800000 782159.158390117\\
12900000 787895.103613871\\
13000000 793628.510256762\\
13100000 799359.400056199\\
13200000 805087.794398072\\
13300000 810813.714325067\\
13400000 816537.180544724\\
13500000 822258.213437236\\
13600000 827976.833063023\\
13700000 833693.059170065\\
13800000 839406.911201027\\
13900000 845118.408300156\\
14000000 850827.569319987\\
14100000 856534.412827841\\
14200000 862238.957112142\\
14300000 867941.220188546\\
14400000 873641.21980589\\
14500000 879338.973451979\\
14600000 885034.498359199\\
14700000 890727.811509982\\
14800000 896418.929642107\\
14900000 902107.869253861\\
15000000 907794.646609055\\
15100000 913479.277741898\\
15200000 919161.778461749\\
15300000 924842.164357726\\
15400000 930520.450803199\\
15500000 936196.652960165\\
15600000 941870.785783499\\
15700000 947542.864025098\\
15800000 953212.902237918\\
15900000 958880.914779896\\
16000000 964546.915817783\\
16100000 970210.919330869\\
16200000 975872.939114617\\
16300000 981532.988784204\\
16400000 987191.081777969\\
16500000 992847.231360779\\
16600000 998501.450627312\\
16700000 1004153.75250525\\
16800000 1009804.1497584\\
16900000 1015452.65498973\\
17000000 1021099.28064438\\
17100000 1026744.03901248\\
17200000 1032386.94223205\\
17300000 1038028.00229173\\
17400000 1043667.23103347\\
17500000 1049304.64015516\\
17600000 1054940.24121321\\
17700000 1060574.04562505\\
17800000 1066206.06467158\\
17900000 1071836.30949956\\
18000000 1077464.79112396\\
18100000 1083091.52043021\\
18200000 1088716.50817649\\
18300000 1094339.76499586\\
18400000 1099961.3013984\\
18500000 1105581.12777332\\
18600000 1111199.25439097\\
18700000 1116815.69140488\\
18800000 1122430.44885362\\
18900000 1128043.53666281\\
19000000 1133654.96464691\\
19100000 1139264.74251109\\
19200000 1144872.87985296\\
19300000 1150479.38616439\\
19400000 1156084.27083317\\
19500000 1161687.54314468\\
19600000 1167289.21228356\\
19700000 1172889.2873353\\
19800000 1178487.77728777\\
19900000 1184084.69103284\\
20000000 1189680.03736781\\
20100000 1195273.82499691\\
20200000 1200866.06253276\\
20300000 1206456.75849776\\
20400000 1212045.92132549\\
20500000 1217633.55936207\\
20600000 1223219.68086747\\
20700000 1228804.29401684\\
20800000 1234387.40690177\\
20900000 1239969.02753157\\
21000000 1245549.16383445\\
21100000 1251127.82365876\\
21200000 1256705.01477416\\
21300000 1262280.7448728\\
21400000 1267855.0215704\\
21500000 1273427.85240742\\
21600000 1278999.24485013\\
21700000 1284569.20629166\\
21800000 1290137.7440531\\
21900000 1295704.86538448\\
22000000 1301270.57746581\\
22100000 1306834.88740805\\
22200000 1312397.80225414\\
22300000 1317959.32897988\\
22400000 1323519.4744949\\
22500000 1329078.24564362\\
22600000 1334635.64920609\\
22700000 1340191.69189891\\
22800000 1345746.38037611\\
22900000 1351299.72122999\\
23000000 1356851.72099195\\
23100000 1362402.38613334\\
23200000 1367951.72306627\\
23300000 1373499.73814439\\
23400000 1379046.43766367\\
23500000 1384591.82786319\\
23600000 1390135.91492587\\
23700000 1395678.70497923\\
23800000 1401220.20409611\\
23900000 1406760.4182954\\
24000000 1412299.35354272\\
24100000 1417837.01575113\\
24200000 1423373.41078181\\
24300000 1428908.5444447\\
24400000 1434442.42249921\\
24500000 1439975.05065482\\
24600000 1445506.43457172\\
24700000 1451036.57986148\\
24800000 1456565.4920876\\
24900000 1462093.17676616\\
25000000 1467619.63936637\\
25100000 1473144.88531122\\
25200000 1478668.91997799\\
25300000 1484191.74869885\\
25400000 1489713.37676141\\
25500000 1495233.80940925\\
25600000 1500753.05184249\\
25700000 1506271.10921829\\
25800000 1511787.98665139\\
25900000 1517303.6892146\\
26000000 1522818.22193934\\
26100000 1528331.58981611\\
26200000 1533843.79779497\\
26300000 1539354.85078605\\
26400000 1544864.75366\\
26500000 1550373.51124848\\
26600000 1555881.12834458\\
26700000 1561387.60970332\\
26800000 1566892.96004205\\
26900000 1572397.18404093\\
27000000 1577900.28634331\\
27100000 1583402.27155622\\
27200000 1588903.1442507\\
27300000 1594402.90896233\\
27400000 1599901.57019151\\
27500000 1605399.13240395\\
27600000 1610895.60003102\\
27700000 1616390.97747017\\
27800000 1621885.26908527\\
27900000 1627378.47920701\\
28000000 1632870.61213328\\
28100000 1638361.67212952\\
28200000 1643851.66342907\\
28300000 1649340.59023355\\
28400000 1654828.45671322\\
28500000 1660315.26700725\\
28600000 1665801.02522417\\
28700000 1671285.73544209\\
28800000 1676769.40170913\\
28900000 1682252.02804369\\
29000000 1687733.61843476\\
29100000 1693214.17684228\\
29200000 1698693.70719742\\
29300000 1704172.2134029\\
29400000 1709649.69933327\\
29500000 1715126.16883526\\
29600000 1720601.62572802\\
29700000 1726076.07380341\\
29800000 1731549.51682635\\
29900000 1737021.95853503\\
30000000 1742493.40264123\\
30100000 1747963.85283058\\
30200000 1753433.31276281\\
30300000 1758901.78607208\\
30400000 1764369.27636718\\
30500000 1769835.7872318\\
30600000 1775301.32222483\\
30700000 1780765.88488056\\
30800000 1786229.47870895\\
30900000 1791692.1071959\\
31000000 1797153.77380344\\
31100000 1802614.48197002\\
31200000 1808074.23511073\\
31300000 1813533.03661751\\
31400000 1818990.88985941\\
31500000 1824447.79818282\\
31600000 1829903.76491166\\
31700000 1835358.79334764\\
31800000 1840812.88677044\\
31900000 1846266.04843798\\
32000000 1851718.28158658\\
32100000 1857169.58943117\\
32200000 1862619.97516556\\
32300000 1868069.44196258\\
32400000 1873517.99297429\\
32500000 1878965.63133222\\
32600000 1884412.36014753\\
32700000 1889858.18251122\\
32800000 1895303.1014943\\
32900000 1900747.12014803\\
33000000 1906190.24150404\\
33100000 1911632.46857457\\
33200000 1917073.80435263\\
33300000 1922514.25181218\\
33400000 1927953.81390832\\
33500000 1933392.49357744\\
33600000 1938830.29373742\\
33700000 1944267.2172878\\
33800000 1949703.26710994\\
33900000 1955138.44606717\\
34000000 1960572.757005\\
34100000 1966006.20275125\\
34200000 1971438.78611622\\
34300000 1976870.50989285\\
34400000 1982301.37685688\\
34500000 1987731.38976699\\
34600000 1993160.551365\\
34700000 1998588.86437596\\
34800000 2004016.33150833\\
34900000 2009442.95545414\\
35000000 2014868.73888913\\
35100000 2020293.68447287\\
35200000 2025717.79484895\\
35300000 2031141.07264508\\
35400000 2036563.52047323\\
35500000 2041985.14092981\\
35600000 2047405.93659577\\
35700000 2052825.91003674\\
35800000 2058245.06380318\\
35900000 2063663.4004305\\
36000000 2069080.92243918\\
36100000 2074497.63233492\\
36200000 2079913.53260876\\
36300000 2085328.62573719\\
36400000 2090742.91418232\\
36500000 2096156.40039193\\
36600000 2101569.08679968\\
36700000 2106980.97582516\\
36800000 2112392.06987403\\
36900000 2117802.37133815\\
37000000 2123211.8825957\\
37100000 2128620.60601127\\
37200000 2134028.54393598\\
37300000 2139435.69870762\\
37400000 2144842.07265074\\
37500000 2150247.66807675\\
37600000 2155652.48728406\\
37700000 2161056.53255814\\
37800000 2166459.80617169\\
37900000 2171862.31038469\\
38000000 2177264.04744453\\
38100000 2182665.01958612\\
38200000 2188065.22903196\\
38300000 2193464.67799228\\
38400000 2198863.36866512\\
38500000 2204261.30323643\\
38600000 2209658.48388017\\
38700000 2215054.91275839\\
38800000 2220450.59202136\\
38900000 2225845.52380765\\
39000000 2231239.7102442\\
39100000 2236633.15344645\\
39200000 2242025.8555184\\
39300000 2247417.81855272\\
39400000 2252809.04463085\\
39500000 2258199.53582306\\
39600000 2263589.29418854\\
39700000 2268978.32177552\\
39800000 2274366.62062132\\
39900000 2279754.19275247\\
40000000 2285141.04018475\\
40100000 2290527.16492332\\
40200000 2295912.56896276\\
40300000 2301297.25428719\\
40400000 2306681.22287032\\
40500000 2312064.47667556\\
40600000 2317447.01765607\\
40700000 2322828.84775485\\
40800000 2328209.96890482\\
40900000 2333590.38302892\\
41000000 2338970.09204012\\
41100000 2344349.09784158\\
41200000 2349727.40232665\\
41300000 2355105.00737899\\
41400000 2360481.91487264\\
41500000 2365858.12667206\\
41600000 2371233.64463224\\
41700000 2376608.47059875\\
41800000 2381982.60640782\\
41900000 2387356.05388638\\
42000000 2392728.81485219\\
42100000 2398100.89111385\\
42200000 2403472.2844709\\
42300000 2408842.99671386\\
42400000 2414213.02962433\\
42500000 2419582.38497502\\
42600000 2424951.06452987\\
42700000 2430319.07004403\\
42800000 2435686.403264\\
42900000 2441053.06592767\\
43000000 2446419.05976435\\
43100000 2451784.38649488\\
43200000 2457149.04783167\\
43300000 2462513.04547876\\
43400000 2467876.38113187\\
43500000 2473239.05647848\\
43600000 2478601.07319787\\
43700000 2483962.43296122\\
43800000 2489323.1374316\\
43900000 2494683.18826408\\
44000000 2500042.58710577\\
44100000 2505401.33559588\\
44200000 2510759.43536576\\
44300000 2516116.88803898\\
44400000 2521473.69523137\\
44500000 2526829.85855108\\
44600000 2532185.37959863\\
44700000 2537540.25996695\\
44800000 2542894.50124148\\
44900000 2548248.10500014\\
45000000 2553601.07281349\\
45100000 2558953.40624467\\
45200000 2564305.10684953\\
45300000 2569656.17617666\\
45400000 2575006.61576741\\
45500000 2580356.42715598\\
45600000 2585705.61186945\\
45700000 2591054.17142783\\
45800000 2596402.1073441\\
45900000 2601749.42112428\\
46000000 2607096.11426746\\
46100000 2612442.18826584\\
46200000 2617787.64460481\\
46300000 2623132.48476295\\
46400000 2628476.71021211\\
46500000 2633820.32241745\\
46600000 2639163.32283746\\
46700000 2644505.71292404\\
46800000 2649847.49412253\\
46900000 2655188.66787174\\
47000000 2660529.235604\\
47100000 2665869.19874524\\
47200000 2671208.55871495\\
47300000 2676547.31692631\\
47400000 2681885.47478619\\
47500000 2687223.03369517\\
47600000 2692559.99504764\\
47700000 2697896.36023177\\
47800000 2703232.13062964\\
47900000 2708567.30761716\\
48000000 2713901.89256424\\
48100000 2719235.88683473\\
48200000 2724569.29178649\\
48300000 2729902.10877147\\
48400000 2735234.33913569\\
48500000 2740565.98421928\\
48600000 2745897.04535658\\
48700000 2751227.5238761\\
48800000 2756557.42110062\\
48900000 2761886.73834718\\
49000000 2767215.47692714\\
49100000 2772543.63814622\\
49200000 2777871.22330451\\
49300000 2783198.23369655\\
49400000 2788524.67061131\\
49500000 2793850.53533226\\
49600000 2799175.82913742\\
49700000 2804500.55329935\\
49800000 2809824.70908521\\
49900000 2815148.29775679\\
50000000 2820471.32057054\\
};
\end{axis}
\end{tikzpicture}% \end{minipage}
	\caption{links: Graph von $\frac{\pi(x) \ln x}x$, rechts: Graphen von $ \pi(x) $(fest) und $ \frac{x}{\ln x} $(punktiert)}
	\label{fig:quotient}
\end{figure}

Für den Beweis der Aussage brauchen wir zunächst einige Hilfssätze. Zunächst zeigen, wir, dass die Riemannsche Zetafunktion keine Nullstellen für $ \Re(z)\ge 1 $ haben kann und folgern schließlich mit Hilfe eines analytischen Arguments den Primzahlensatz.

Wir definieren uns an dieser Stelle die drei Funktionen, die wir im Verlauf des Beweises benötigen. Vermöge
\begin{alignat*}{2}
\zeta(z)&:=\sum_{n\in \N} \frac{1}{n^z},&\qquad &z \in \C \text{ und } \Re(z)>1; \\
\theta(x)&:=\sum_{p\in \mathbb P, \; p\le x} \ln p,&\qquad &x\in \R; \\
\Phi(z)&:= \sum_{p\in \mathbb P} \frac{\ln p}{p^z},&\qquad &z\in \C \text{ und } \Re(z)>1.
\end{alignat*}

\begin{lem}[Fortsetzung der Riemannschen Zetafunktion]\label{pol}
Die \emph{Riemannsche $ \zeta $-Funktion} ist holomorph fortsetzbar auf $ \{z\in \C| \Re(z)>0\} \setminus\{1\} $ und besitzt bei $ z=1 $ eine einfache Polstelle.
\end{lem}
\begin{proof}[Übung]
\end{proof}
\begin{lem}[Residuum an der Stelle $ x=1 $]
$ \Res(\zeta, 1)=1 $. Insbesondere ist die Funktion $ \zeta(z)-\frac{1}{z-1} $ holomorph Fortsetzbar auf $ \{z \in \C | \Re(z)>0\} $.
\end{lem}
\begin{proof}
mit dem Cauchy-Integralkriterium für Reihen folgt für $ x>1 $.
\[
 \frac{1}{x-1}=\int_{1}^\infty \frac{1}{t^x} \, \mathrm dt \le \underbrace{\sum_{n=1}^\infty \frac{1}{n^x}}_{\zeta(x)} \le 1+\int_{1}^\infty \frac{1}{t^x} \,\mathrm dt = 1+ \frac{1}{x-1}.
\]
Daraus folgt
\[
1\le (x-1) \zeta(x) \le (x-1)+1.
\]
Gehen wir zum Grenzwert $ x\to 1+ $ über, so ergibt sich mit Lemma \ref{pol}
\[
\Res(\zeta, 1)=\lim_{x\to 1+} (x-1)\zeta(x)=1.
\]
\end{proof}
\begin{lem}[Eulersche Produktformel] \label{euler}
Für $ \Re(z)>1 $ gilt
\[
\zeta(z)=\prod_{p\in \mathbb P} (1- p^{-z})^{-1}.
\]
Insbesondere besitzt die $ \zeta $-Funktion keine Nullstellen für $ \Re(z) >1$.
\end{lem}

\begin{lem} \label{linear}
$ \theta(x)=\mathcal O(x) $.
\end{lem}

\begin{proof}
Sei $ n\in \N $. Mit dem Binomiallehrsatz folgt:
\[
2^{2n} = (1+1)^{2n}=\sum_{k=0}^{2n} \begin{pmatrix} 2n \\ k \end{pmatrix} \ge \begin{pmatrix} 2n \\ n \end{pmatrix}.
\]
Es ist 
\[
\begin{pmatrix}2n \\ n \end{pmatrix} = \frac{(2n)!}{(n!)^2}\ge \prod_{p \in \mathbb P, \, n<p\le 2n} p.
\]
% Ist diese Begrüngung wirklich nötig?
Da $ \begin{psmallmatrix} 2n \\ n \end{psmallmatrix} \in \N $, folgt: $ (n!)^2$ teilt $ (2n)! $. Außerdem gilt auch: $ \prod_{p \in \mathbb P, \, n<p\le 2n} p$ teilt $(2n)!$. Da die beiden Teiler teilerfremd sind, gilt insbesondere auch $ (n!)^2 \prod_{p \in \mathbb P, \, n<p\le 2n} p$ teilt $(2n)! $ und es folgt die Ungleichung.

Es ist
\[
2^{2n}\ge \prod_{\substack{p\in \P \\ n<p\le 2n}} p =e^{\sum\limits_{p\in \P, n< p \le 2n} \ln p}=e^{\theta(2n)-\theta(n)}.
\]
Nach dem Logarithmieren folgt
\[
\theta(2n)-\theta(n)\le 2n \ln 2.
\]
Wir betrachten im Folgenden für $ m\in \N $
\begin{align*}
\theta(2^m)&=\theta(2^m)-\underbrace{\theta(2^0)}_{=0}=\sum_{n=1}^m (\theta(2^n)-\theta(2^{n-1}) \\ 
&\le \sum_{n=1}^m 2^n \ln 2 = (2^{m+1}-2)\ln 2< 2^{m+1} \ln 2.
\end{align*}
Für alle $ x\ge 1 $ können wir $m$ derart wählen, dass $ 2^{m-1} \le x < 2^m $. Es folgt
\[
\theta(x) \le \theta(2^m) \le 2^{m+1}\ln 2 = (4 \ln 2)2^{m-1} \le (4 \ln 2) x.
\]
Für $ x\le 1 $ ist $ \theta(x)=0 $ und damit folgt insgesamt $ \theta(x)=\mathcal O(x) $.
\end{proof}

\begin{lem}[Äquivalente Bedingung des Primzahlensatzes] \label{equiv}
$ \theta(x) \sim x \iff \pi(x) \sim \frac{x}{\ln x} $.
\end{lem}
\begin{proof}
Für $ x\ge 1 $ gilt
\[
0 \le \theta(x) = \sum_{p \in \mathbb P, \; p\le x} \underbrace{\ln p}_{\le \ln x}\le \sum_{p \in \mathbb P, \; p\le x} \ln x = \pi(x) \ln x.
\]
Es folgt
\[
\frac{\theta(x)}{x} \le \pi(x) \frac{\ln x}{x}.
\]
Sei $ \epsilon >0 $ beliebig. Dann gilt desweiteren
\begin{align*}
\theta(x) &\ge \sum_{p\in \mathbb P, \; x^{1-\eps}<p \le x} \ln p \ge  \sum_{p \in \mathbb P, \, x^{1-\eps} <p\le x} \ln (x^{1-\eps})\\
&= (1-\eps) \ln(x) \Big(\pi(x)-\pi(x^{1-\eps})\Big)\ge (1-\eps) \ln(x) \Big(\pi(x)- x^{1-\eps}\Big).
\end{align*}
Dann folgt insbesondere auch schon % Vielleicht ein Zwischenschritt?
\[
0 \le \pi(x) \frac{\ln x}{x}- \frac{\theta(x)}{x} \le \frac{\eps}{1-\eps} \frac{\theta(x)}{x} + \frac{\ln(x)}{x^{\eps}}.
\]
Für hinreichend große $ x $ folgt nach Lemma \ref{linear} die Existenz von $ c\in \R_+ $, sodass für alle $ x\ge 0 $ $ \frac{\theta(x)}{x}\le c  $. Wähle $ \eps:= \frac{\tilde \eps}{2c+\tilde \eps} $. Sei $ x $ desweiteren hinreichend groß, sodass für fest gewähltes $ \eps $
\[ \frac{\ln x}{x^\eps} < \frac{\tilde \eps}{2} \]
gilt. Insgesamt folgt also
\[
\Big |\pi(x) \frac{\ln x}{x}- \frac{\theta(x)}{x}\Big | < \frac{\tilde \eps}{2} + \frac{\tilde \eps}{2}= \tilde \eps. 
\]
Schließlich folgt $ \pi(x) \sim \frac{x}{\ln x} \iff \theta(x) \sim x $.
\end{proof}

\begin{lem} \label{hol}
$ \zeta(z)\neq 0 $ für $ \Re(z) \ge 1 $ und $ \Phi(z)-\frac{1}{z-1} $ ist holomorph fortsetzbar auf einer Obermenge von $ \{z\in \C| \Re(z)\ge 1\} $.
\end{lem}

\begin{proof}
\begin{enumerate}[1)]
\item Wohldefiniertheit und Holomorphie von $ \Phi $:

Hierfür zeigen wir die kompakte und absolute Konvergenz der Reihe 
$\sum_{n=1}^\infty \frac{\ln n}{n^z}$. Um dies zu zeigen, suchen wir eine Majorante für $ \Re(z)\ge 1+\delta $, wobei $ \delta>0 $ beliebig gewählt ist. Dann folgt die gleichmäßige Konvergenz nach dem Weierstraßschen M-Test. Jede kompakte Teilmenge ist in einer der Streifen zu finden und es folgt die Aussage (nach Weierstraß nimmt der Realteil sein Minimum an). Für hinreichend großes $ n_0 $ folgt mit $ \frac{\ln n}{n^{\delta/2}} \stackrel{n\to \infty}\to 0 $
\[
\sum_{n=n_0}^\infty \Big |\frac{\ln n}{n^z} \Big |  = \sum_{n=n_0}^\infty  \frac{\ln n}{n^{\Re(z)}} \le \sum_{n=n_0}^\infty \frac{n^{\delta/2}}{n^{\Re(z)}}\le \sum_{n=n_0}^\infty \frac{1}{n^{1+\delta/2}} < \infty.
\]
\item Konstruktion einer meromorphen Fortsetzung:

Für $ \Re(z)>1 $ gilt mit der Eulerschen Produktformel
\[
\zeta(z)=\prod_{p\in \P} (1-p^{-z})^{-1}.
\]
Auf einer geeigneten Umgebung lässt sich der komplexe Logarithmus definieren. Anwenden des Logarithmus auf die Eulersche Produktformel ergibt:
\[
\ln \zeta(z) = \ln \prod_{p\in \P}(1-p^{-z})^{-1}=- \sum_{p\in \P}\ln(1-p^{-z})
\]
Differenzieren auf beiden Seiten ergibt
\begin{align*}
(\ln(\zeta(z))'&= - \Bigg (\sum_{p\in \P}\ln(1-p^{-z})\Bigg )'\\
-\frac{\zeta'(z)}{\zeta(z)}&= \sum_{p\in \P} \frac{p^{-z} \ln p}{1-p^{-z}}=\sum_{p\in \P} \frac{\ln p}{p^z-1}.
\end{align*}
Mit der Identität 
$\frac{1}{p^z-1}=\frac{1}{p^z} + \frac{1}{p^z(p^z-1)}$ 
ergibt sich
\[
-\frac{\zeta'(z)}{\zeta(z)}= \Phi(z)+ \underbrace{\sum_{p\in \P} \frac{\ln p}{p^z(p^z-1)}}_{=:H(z)}.
\]

Sei $ \Re(z) \ge \frac{1}{2} + \delta $ für $ \delta>0 $ beliebig. Für hinreichend große $ p $ folgt ähnlich wie oben
\[
\Big | \frac{\ln p}{p^z(p^z-1)}\Big | \le \frac{p^{\delta}}{p^{1/2+\delta}\cdot \frac{1}{2} p^{1/2+\delta}}= 2 \frac{1}{p^{1+\delta}}.
\] % Hier würde ich einen Zwischenschritt machen bei der ersten Ungleichung

Es folgt die kompakten und absoluten Konvergenz (s.o.) von $ \sum_{p\in \P} \frac{\ln p}{p^z(p^z-1)} $ für $ \Re(z)>\frac{1}{2} $. Damit folgt, dass $ H(z) $ wohldefiniert und holomorph ist. Insbesondere haben wir eine meromorphe Fortsetzung $ \Phi(z) = -\frac{\zeta'(z)}{\zeta(z)} - H(z)$ gefunden, wobei die Singularitäten gerade bei den Null- und Polstellen von $ \zeta(z) $ liegen (Die Nullstellen einer von 0 verscheidenen Funktion innehrhalb eines Gebiets sind isoliert). 

Wir zeigen nun die Holomorphie von $ \Phi(z)-\frac{1}{z-1} $ im Punkt $ z=1 $ (die einzige Polstelle von $ \zeta $). Nach Lemma \ref{pol} folgt, dass es holomorphe Funktionen $ H_1, H_2 $ gibt mit 
\[
\zeta(z)=\frac{1}{z-1}+ H_1(z) \text{ und } \zeta'(z)=- \frac{1}{(z-1)^2}+ \underbrace{H_2(z)}_{:=H_1'(z)}
\]
für $ \Re(z)>\frac{1}{2} $. Es gilt
\[
\frac{\zeta'(z)}{\zeta(z)}+ \frac{1}{z-1}= \frac{(z-1) \zeta'(z)+\zeta(z)}{(z-1)\zeta(z)}= \frac{H_1(z)+(z-1)H_2(z)}{1+(z-1) H_1(z)}.
\]
Offensichtlich ist $\tilde H(z):= \Phi(z) - \frac{1}{z-1} $ damit in $ z=1 $ stetig ergänzbar bzw. mit dem Riemannschen Hebbarkeitssatz folgt, dass $ \tilde H(z)= \Phi(z) - \frac{1}{z-1} $ auch holomorph in $ z=1 $ ergänzt werden kann. % Riemann reicht doch, vlt mit einer kurzen Begründung für die Beschränktheit
\item $ \zeta(z)\neq 0 $, für $ \Re(z)\ge 1, z\neq 1 $
Nach Lemma \ref{euler} genügt es potentielle Nullstellen für $ \Re(z)=1 $ zu untersuchen.
Mit dieser Aussage wäre gezeigt, dass $ \Phi(z)-\frac{1}{z-1} $ holomorph auf einer Obermenge von $ \{z| \Re(z) \ge 1\} $ ist.

Zunächst merken wir an, dass
\[
\lim_{\eps \to 0+} \eps \Phi(1+\eps)= \lim_{\eps \to 0+} \eps\Big(\tilde H(1+\eps)+\tfrac{1}{\eps}\Big)=1.
\]
Sei $\alpha \in \R\setminus\{0\}$, $ z_1=1+i\alpha $ Nullstelle mit Ordnung $ \mu\in \N_0 $ und $ z_2=1+2i\alpha $ Nullstelle mit Ordnung $\nu\in \N_0$. 
\begin{seg}[Sei $ \mu=0 $]
\[
\lim_{\eps\to 0+} \eps \Phi(1+\eps+i\alpha)=0
\]
\end{seg}
\begin{seg}[Sei $ \mu\neq 0 $]
Es existiert nach Definition von $ \mu $ $ h_1, h_2 $ holomorph mit
\begin{align*}
\zeta(z)&=c (z-1-i\alpha)^\mu + (z-1-i\alpha)^{\mu+1} h_1(z)\\
\zeta'(z)&= \mu c (z-1-ia)^{\mu-1} + (s-1-i\alpha)^{\mu} h_2(z)
\end{align*}
Es folgt
\[
-\eps \frac{\zeta'(1+\eps+i\alpha)}{\zeta(1+\eps+i\alpha)}=-\eps \frac{\mu c \eps^{\mu-1} + \eps^{\mu} h_2(1+\eps+i\alpha)}{c\eps^{\mu} + \eps^{\mu+1} h_1(1+\eps+i\alpha)} =- \frac{ \mu c+ \eps h_2(1+\eps+i\alpha)}{c + \eps h_2(1+\eps+i\alpha)}.
\]
\end{seg}
Und damit
\[
\lim_{\eps\to 0+}\eps \Phi(1+\eps+i\alpha)= \lim_{\eps\to 0+}- \eps\frac{\zeta'(1+\eps+i\alpha)}{\zeta(1+\eps+i\alpha)} =-\mu.
\]
Völlig analog folgt auch
\[
\lim_{\eps\to 0+} \eps \Phi(1+\eps+2i\alpha)=-\nu.
\]
Es ist für $ \Re(z)>1 $
\[
\overline{\zeta(z)}=\overline{\sum_{n=1}^\infty n^{-z}}= \sum_{n=1}^\infty n^{-\overline z} = \zeta(\overline z).
\] 
Und damit folgt schließlich auch
\[
\lim_{\eps\to 0+} \eps \Phi(1+\eps-i\alpha)= \lim_{\eps\to 0+}- \eps \overline{\frac{\zeta'(z_1)}{\zeta(z_1)}}  =-\mu \text{ und } \lim_{\eps\to 0+} \eps \Phi(1+\eps-2i\alpha)=-\nu.
\]
Wir betrachten nun für $ \eps>0 $
\begin{align*}
\sum_{r=-2}^2 \begin{pmatrix} 4 \\ 2+r \end{pmatrix} \eps \Phi(1+\eps+ir\alpha)&=\eps \sum_{p\in \P} \frac{\ln p}{p^{1+\eps}} \sum_{r=-2}^2 \begin{pmatrix} 4 \\ 2+r \end{pmatrix} p^{-ir\alpha} \\ 
&= \eps \sum_{p\in \P} \frac{\ln p}{p^{1+\eps}} \sum_{r=0}^4 \begin{pmatrix} 4 \\ r \end{pmatrix} \Big(p^{-i\frac{\alpha}{2}}\Big)^r \Big ( p^{i\frac \alpha 2}\Big )^{4-r}\\
&= \eps \sum_{p\in \P} \frac{\ln p}{p^{1+\eps}} \Big(\underbrace{p^{-i\frac\alpha 2} + p^{i \frac \alpha 2}}_{=2\Re\Big(p^{i \frac \alpha 2}\Big)}\Big)^4\ge 0.
\end{align*}
Gehen wir zum Grenzwert $ \eps \to 0+ $ über, so erhalten wir die Ungleichung
\[
-\nu - 4\mu + 6 - 4\mu - \nu = 6-8 \mu - 2\nu \ge 0,
\]
und es folgt $ \mu=0 $.
\end{enumerate}
\end{proof}
\begin{st}[Analytisches Argument] \label{ana}
Sei $ f:[0, \infty) \to \R $ beschränkt und messbar. Weiter sei $ g:\{z\in \C|\Re z >0\} \to \C, z \mapsto \int_{0}^\infty f(t) e^{-zt}\dx[t] $ holomorph. Zudem gebe es eine holomorphe Fortsetzung von $ g $ auf einer offene Obermenge von $ \{z \in \C| \Re(z) \ge 0\} $. Dann existiert $ \lim_{T\to \infty} \int_0^T f(t) \dx[t] $, und der Grenzwert hat den Wert $ g(0) $.
\end{st}
\begin{proof}
Für $ T>0 $ setzen wir $ g_T: \C \to \C $ als $ g_T(z):= \int_{0}^T f(t) e^{-zt}\dx[t] $. Wegen der Beschränktheit und Messbarkeit von $ f $ ist $ g_T $ wohldefiniert. $ g_T $ ist eine ganze Funktion, denn betrachten wir den Differentialquotienten
\[
\Big | \frac{g_T(z+h)-g_T(z)}{h}+ \int_0^T t f(t) e^{-zt}\dx[t]\Big |\le \int_0^T |f(t) e^{-zt}| \Big | \frac{e^{-ht}-1+ht}h \Big | \dx[t].
\]
Setze $ F(x):= e^{-zht} $. Es ist
\begin{align*}
e^{-ht}-1+ht &= F(1)-F(0)-F'(0)=\int_0^1 [F'(x)-F'(0)]\dx= \int_0^1 \int_0^x F''(y)\dx[y] \dx\\
\Big | \frac{e^{-ht}-1+ht}h \Big | &\le \int_0^1 \int_0^x |ht^2 e^{-yht}|\dx[y]\dx\le |h|T^2 e^{|h| T} \frac{T^2}{2}\stackrel{h\to 0}\to 0.
\end{align*}
Und damit folgt die Holomorphie von $ g_T $ auf ganz $ \C $. Sei $ R>0 $ beliebig und $ C $ der orientierte Rand des konvexen (insbesondere einfach zusammenhängenden) berandeten Gebiets $ \{z  \in \C | |z|\le R, \Re(z) \ge - \delta\} $, wobei $ \delta>0 $ derart gewählt ist, dass $ g $ holomorph im berandeten Gebiet ist. Es folgt leicht mittels Anwendung von Heine-Borel auf der Strecke $ \overline{-Ri, Ri} $, dass wir ein $ \delta:=\delta(R)>0 $ finden, sodass $ g $ auf $ \{ z \in \C | |z|\le R, \Re(z) > - 2 \delta\} $ holomorph ist. Betrachte hierfür eine offene Überdeckung von $ \overline{-Ri, Ri} $ aus offenen Quadraten der Form $ \{z\in \C| |\Re(z-x)|<\delta \text{ und } |\Im(z-x)|<\delta_x\} $ auf denen $ g $ holomorph ist. Dann findet sich eine endliche Teilüberdeckung $ \{Q_1,..., Q_N\} $ und sei $ a $ die kleinsten Seitenlänge  aus unserer endlichen Teilüberdeckung, dann wähle $ \delta:= \frac{a}{4} $.

Setze $ G(z):=(g(z)-g_T(z))e^{zT} (1+ \frac{z^2}{R^2}) $. Mit der Cauchyschen Integralformel folgt
\[
g(0)-g_{T}(0)=G(0)=\frac{1}{2\pi i} \int_C (g(z)-g_T(z)) e^{zT} (1+\frac {z^2} {R^2})\frac{\dx[z]}{z}.
\]
Wir spalten im Folgenden den Integrationsweg in zwei Teilwege
\[
C_+:= C \cap \{z\in \C| \Re(z)>0\} \text{ und } C_-:= C\cap \{z\in \C|\Re(z)<0\}
\]
auf. Es ist
\begin{align*}
|g(0)-g_T(0)|\le \Bigg|\frac{1}{2\pi i} \int_{C_+} (g(z)&-g_T(z)) e^{zT} (1+\frac {z^2} {R^2})\frac{\dx[z]}{z}\Bigg|\\ &+ \Bigg| \frac{1}{2\pi i} \int_{C_-} g(z) e^{zT} (1+\frac {z^2} {R^2})\frac{\dx[z]}{z} \Bigg|+ \Bigg| \frac{1}{2\pi i} \int_{C_-} g_T(z) e^{zT} (1+\frac {z^2} {R^2})\frac{\dx[z]}{z} \Bigg|.
\end{align*}
Setze $ B:=\sup\{|f(t)|\, | t\ge 0\} $. Es folgt für den Integrationsweg $ C_+ $
\[
|g(z)-g_T(z)|=\Bigg | \int_T^\infty f(t) e^{-zt}\dx[t]\Bigg |\le B \int_T^\infty |e^{-zt}| \dx[t]=\frac{Be^{-\Re(z) T}}{\Re(z)}.
\]
Weiterhin gilt
\begin{align*}
\Big|e^{zT} (1+\frac{z^2}{R^2}) \frac{1}{z}\Big |= \frac{e^{\Re(z)T}}{R^2} |(z^2+R^2)\cdot \frac{1}{z}| \stackrel{R^2=z\bar z}=  \frac{e^{\Re(z)T}}{R^2} \cdot |z+\bar z|=\frac{e^{\Re(z)T}}{R^2}\cdot 2 \Re(z).
\end{align*}
Wir können also abschätzen
\[
\Bigg | \frac{1}{2\pi i} \int_{C_+} (g(z)-g_T(z))e^{zT} (1+\frac{z^2}{R^2}) \frac{\dx[z]}{z}\Bigg| \le \frac{\pi R}{2\pi} \cdot \frac{e^{\Re(z) T}}{R^2} \cdot 2 \Re(z) \cdot \frac{Be^{-\Re(z) T}}{\Re(z)}= \frac{B}{R}.
\]
Für $ C_- $ betrachten wir die Integrale für $ g $ und $ g_T $ seperat. Sei $ \tilde C_-=\{z\in \C||z|=R, \Re(z)<0\} $. Da $ g_T $ ganz ist, können wir den Integrationweg $ C_- $ durch diesen ersetzen. Dann folgt
\[
|g_T(z)|=\Bigg|\int_0^T f(t) e^{-zt} \dx[t]\Bigg|\le B \int_0^T e^{- \Re(z)t} \dx[t]\le \frac{Be^{-\Re(z)T}}{|\Re(z)|}.
\]
Ähnlich wie oben folgt auch 
 \[
 \Big |e^{zT}(1+\frac{z^2}{R^2})\frac{1}{z}\Big |=...= \frac{e^{\Re(z) T}}{R^2} \cdot 2 |\Re(z)|.
 \]
Schließlich,
\[
\Bigg | \frac{1}{2\pi i} \int_{\tilde C_-} (g(z)-g_T(z))e^{zT} (1+\frac{z^2}{R^2}) \frac{\dx[z]}{z}\Bigg| \le ... \le \frac{B}{R}.
\]
Wir können also $ R $ so groß wählen, dass $ \frac{B}{R}< \frac{\eps}{4} $. Das verbleibende Integral wird für genügend kleines $ t $ und dazu passend hinreichend große $T$ beliebig klein. Um dies zu zeigen, wählen wir $ 0<t\le \delta $ beliebig. Setze $ S:=\sup\{|g(z)|\Big|1+\frac{z^2}{R^2}\Big|\frac{1}{|z|}, z \in C_-\} $. Dann folgt mit hinreichend großem $ T $
\begin{align*}
\Bigg |\frac{1}{2\pi i} \int_{C_-}g(z) e^{zT} \Big(1+\frac{z^2}{R^2}\Big) \frac{\dx[z]}{|z|} \Bigg|&\le \frac{S}{2\pi} \cdot\Bigg ( \Bigg | \int_{C_- \cap \{z: \Re{z}\le t\}} \underbrace{e^{zT}}_{|\cdot|\le e^{tT}}\dx[z]\Bigg |+ \Bigg | \int_{C_- \cap \{z: \Re{z}\ge t\}} \underbrace{e^{zT}}_{|\cdot|\le 1}\dx[z]\Bigg | \Bigg)\\
&\le   \frac{S}{2\pi} ( L(C_- \cap \{z: \Re{z}\le t\}) e^{tT}+ L(C_- \cap \{z: \Re{z}\ge t\}) )\\
&\le   \frac{S}{2\pi} ( \underbrace{L(C_- \cap \{z: \Re{z}\le t\}) e^{tT}}_{\le \frac{\pi\eps}{2S}} + \underbrace{R+t-\sqrt{R^2-t^2}}_{\le \frac{\pi\eps}{2S}}\\
&= \frac{\eps}{2}.
\end{align*}
Insgesamt folgt schließlich für hinreichend großes T und $ \eps>0 $
\[
|g(0)-g_T(0)|\le \eps.
\]
\end{proof}
\begin{lem} \label{int}
$ \int_1^\infty \frac{\theta(x)-x}{x^2}\dx $ existiert.
\end{lem}
\begin{proof}
Für $ \Re(z)>1 $ folgt mit partieller Summation
\begin{align*}
\Phi(z)&=\sum_{p\in \P} \frac{\ln p}{p^z}= \lim_{x\to \infty} \sum_{p\in \P, \, p\le x} \frac{\ln p}{p^z}\\
 &= \lim_{x\to \infty} \underbrace{\frac{\theta(x)}{x^z}}_{\stackrel{\text{Lem. \ref{linear}}}\to 0}+ z \int_{2}^\infty \frac{\theta(x)}{x^{z+1}} \dx \\
 &=z \int_{1}^\infty \frac{\theta(x)}{x^{z+1}}\dx \stackrel{x=e^t}= z \int_0^\infty e^{-zt} \theta(e^t) \dx[t].
\end{align*}
Sei $ f(t):= \theta(e^t) e^{-t} -1  $. $ f $ ist nach Lemma \ref{linear} beschränkt und messbar.
Vermöge 
\begin{align*}
g(z)&:= \int_0^\infty (\theta(e^t)e^{-t}-1)e^{-zt}\dx[t]\\
&= \underbrace{\int_{0}^\infty \theta(e^t) e^{-(z+1)t}}_{=\frac{\Phi(z+1)}{z+1}}- \underbrace{\int_0^\infty e^{-zt}\dx[t]}_{\frac{1}{z}}\\
&= \frac{\Phi(z+1)-\frac{1}{z}}{z+1} + \underbrace{\frac{1}{z(z+1)}- \frac{1}{z}}_{=-\frac{1}{z+1}}.
\end{align*}
Damit ist $ g $ wohldefiniert und holomorph auf einer Obermenge von $ \{z\in \C: \Re(z)\ge 0\} $ nach Lemma \ref{hol}. Wir können also Satz \ref{ana} anwenden. Dann existiert
\[
\int_0^\infty (\theta(e^t) e^{-t}-1) \dx[t]\stackrel{x=e^t}= \int_1^\infty \frac{\theta(x) x^{-1} -1}{x}\dx= \int_1^\infty \frac{\theta(x)-x}{x^2}\dx.
\]
\end{proof}
\begin{lem} \label{final}
$ \theta(x)\sim x $.
\end{lem}
\begin{proof}
Angenommen für ein $ \lambda>1 $ existiert hinreichend großes $ x $ mit $ \theta(x) \ge \lambda x $. Da $ \theta(x) $ offensichtlich monoton steigend folgt
\[
\int_x^{\lambda x} \frac{\theta(t)-t}{t^2} \dx[t] \ge \int_x^{\lambda x} \frac{\lambda x- t}{t^2} \dx[t]= \int_1^\lambda \frac{\lambda-t}{t^2} \dx[t] >0
\]
Und nach dem Cauchy-Kriterium divergiert das Integral $ \int_1^\infty \frac{\theta(t)-t}{t^2} \dx[t] $. Es ergibt sich der Widerspruch zu Lemma \ref{int} und es folgt $ \lim\sup_{x\to \infty} \frac{\theta(x)}{x}\le 1 $. Sehr ähnlich lässt sich zeigen, dass auch die Anname $ \theta(x)\le \lambda x $ mit $ \lambda < 1 $ für hinreichend große $ x $ zum Widerspruch führt, denn
\[
\int_{\lambda x}^x \frac{\theta(t)-t}{t^2}\dx[t] \le \int_{\lambda x}^x \frac{\lambda x-1}{t^2} \dx[t]=\int_{\lambda}^1 \frac{\lambda-t}{t^2}\dx[t]<0.
\]
Es folgt $ \lim\inf_{x\to \infty} \frac{\theta(x)}{x} \ge 1 $. Insgesamt folgt $ \lim_{x\to \infty} \frac{\theta(x)}{x} =1 $.
\end{proof}
Mit Lemma \ref{final} und Lemma \ref{equiv} folgt der Primzahlensatz.
\newpage
\section*{Verwendete Sätze}
An dieser Stelle, die Formulierung einiger wichtiger Sätze. Sei stets $ \K \in \{\R, \C\} $.

\begin{st*}[Cauchy-Integralkriterium]
Es sei $ f $ eine monoton fallende Funktion, die auf dem Intervall $ [p, \infty) $ mit einer ganzen Zahl $ p $ definiert ist und nur positive Werte annimmt. Dann konvergiert die Reihe $ \sum_{n=p}^\infty f(n) $ genau dann, wenn das Integral $ \int_p^\infty f(x) \dx $ existiert. Es gelten die Abschätzungen
\[
\sum_{n=p+1}^\infty f(n) \le \int_p^\infty f(x) \dx \le \sum_{n=p}^\infty f(n).
\]
\end{st*}

\begin{st*}[Residuumberechnung]
Hat $ f $ in $ a $ eine Polstelle 1. Ordnung, gilt: 
\[
\Res_a(f)=\lim_{z\to a}(z-a) f(z)
\]
\end{st*}

\begin{st*}[Binomischer Lehrsatz]
Für alle $ x,y\in \K $ und für alle $ n\in \N $
\[
(x+y)^n= \sum_{k=0}^n \begin{pmatrix} n \\ k \end{pmatrix} x^{n-k} y^k,
\]
wobei $ \begin{psmallmatrix} n\\ k \end{psmallmatrix}= \frac{n!}{k! (n-k)!} $, der Binomialkoeffizient ist.
\end{st*}

\begin{st*}[Kompakte Konvergenz holomorpher Funktionen]
Eine Folge von Funktionen $ (f_n), f_n: O \to \C $ heißt \emph{kompakt konvergent} gegen $ f $, falls für jede kompakte Teilmenge $ K\subset O $ die Funktion $ f_n|_K $ gleichmäßig gegen $ f|_K $ konvergiert. Ist  $ f_n $ holomorph und konvergiert $ f_n $ kompakt gegen $ f $, so ist $ f $ holomorph.
\end{st*}

\begin{st*}[Weierstraßscher M-Test]
Ist $ \sum_{a_n} $ mit $ a_n \ge 0 $ konvergent und gilt $ f_n: M \to \C, |f_n(z)|\le a_n $ auf $ M\subset \C $, so ist die Reihe $ \sum_{n=0}^\infty f_n(z) $ gleichmäßig konvergent auf $ M $ und absolut konvergent für $ z\in M $.
\end{st*}

\begin{st*}[Riemannscher Hebbarkeitssatz]
Sei $ O\subset \C $, $ z_0\in O $, $ f: O \setminus\{z_0\} \to \C $ differenzierbar, und
\[
\exists r>0 \exists M>0: |f(z)|\le M, \quad \text{ für } 0 <|z-z_0|<r
\]
Dann kann $ f $ in $ z=z_0 $ holomorph ergänzt werden.
\end{st*}

\begin{st*}[messbare beschränkte Funktionen sind integrierbar]
Ist $ f $ messbar und Lebesgue-beschränkt, so ist $ f $ integrierbar.
\end{st*}

\begin{st*}[Cauchy-Integralformel]
Sei $ G $ eine einfach zusammenhängendes Gebiet und $ \gamma $ ein stückweise stetig differenzierbarer, geschlossener und doppelpunktfreier Weg in $ G $, der im "`Gegenuhrzeigersinn"' orientiert ist. Weiter sei $ z $ im Inneren von $ \gamma $ und $ f $ holomorph auf $ G $. Dann gilt
\[
f(z)=\frac{1}{2\pi i} \int_\gamma \frac{f(\zeta)}{\zeta-z}\dx[\zeta].
\]
\end{st*}

\begin{st*}[Cauchy-Kriterium für uneigentliche Integrale]
Das uneigentliche Integral $ \int_r^\infty f(x) \dx $ ist genau dann konvergent, wenn
\[
\forall_{\eps>0}\exists_{c>0}\forall_{x', x''\ge c} \Bigg | \int_{x'}^{x''} f(x) \dx \Bigg | < \eps
\]
\end{st*}

Die meisten anderen Sätze, die verwendet worden sind, sollten sich aus dem Kontext ergeben.

\end{document}
